\documentclass[12pt,a4paper,spanish]{article}
\usepackage[spanish]{babel}
\usepackage [T1]{fontenc}
\usepackage [latin1]{inputenc}
\usepackage{graphicx}
\usepackage{float}
\usepackage{array}
	  \oddsidemargin 0in
      \textwidth 6.75in
      \topmargin 0in
      \textheight 10.0in
      \parindent 0em
      \parskip 2ex
\usepackage{anysize}
\marginsize{3cm}{2cm}{1.0cm}{1.0cm}

\pagestyle{plain}

\begin{document}


\title{Minuta de Runi\'on}
\maketitle

\section{Participantes}
Fern\'{a}ndez Nicol\'{a}s


\section{Entrevistado}
Zubiria Alejandro, Responsable de Sistemas

\section{Lugar y Fecha}
Suipacha 1067 1008 Ciudad Aut�noma de Buenos Aires, Lunes 26 de Octubre de 2009

\section{Temas Tratados}
Se trataron temas pertinentes al manejo del \'area y de la empresa en general, considerando que sobre sistemas se maneja toda la informaci\'on de la misma. 
Adem\'as se discuti\'o sobre los problemas de la empresa y los productos principales que se estan vendiendo y la mirada critica sobre la misma.
Se observo y se charlo la evoluci\'on en los \'ultimos tiempos de la empresa y de los negocios que se hicieron.

\begin{itemize}
\item Pregunta nro 1:
	Lic. en Sistemas. 
	MCSE de Microsoft
	Corresponde con las competencias requeridas, incluso las sobrepasa.
\item Pregunta nro 2:
	Consultoria en infraestructura de redes.
	Soporte tecnico.
	Administracion de servidores y redes.
\item Pregunta nro3:
Responsable del area de sistemas. En si es el manejo de infraestructura tecnologica y el cumplimiento de los procesos del software de gestion.
\item Pregunta nro4:
Responder desde la tecnologia a las necesidades de negocio de la empresa.
Mantener el correcto nivel de servicio de los recursos tecnologicos y comunicacionales.
Establecer los adecuados niveles de soporte a los usuarios, tanto de forma local como remota.
Trazar un plan de actualizacion tecnologica a mediano y largo plazo.
\item Pregunta nro5:
El director general.
\item Prgunta nro7:
Al director General y a el sector tecnologico de Marsans Espana.
\item Prgunta nro8:
Actualmente ninguno.
\item Prgunta nro9:
Si, se delegan. Por ejemplo sobre los data entry del area de producto, sobre el chequeo de la administracion de la base de datos.
Ademas se cuenta con la delegacion del soporte de software de gestion a la empresa que lo programo.
Se delega la parte de comunicacion a una persona terciarizada.
\item Prgunta nro10:
Si, suelen ocurrir requerimientos particulares de forma repetitiva para los cuales existen procedimientos de respuesta, tales como bases de conocimientos, manuales de procedimientos o tambien acuerdo de niveles de servicio.
\item Prgunta nro11:
Depende de que decisiones se estan analizando, pero en particular desde la tecnologia el analisis de los procesos de la empresa, la estimacion de crecimiento, la estrategia comercial definida por la empresa.
\item Prgunta nro12:
Al no ser una empresa netamente tecnologica el reporte esta orientado a inversion vs resultado, entonces debe justificar y reportar de alguna forma la influencia que tuvo el presupuesto establecido para el area sobre los procesos de negocio de la empresa.
\item Prgunta nro13:
La relacion en general es coordinal y netamente de trabajo, establecida via mails, telefono y reuniones semanales.
Es limitada a requerimientos especificos que sean de influencia por las dos areas.
\item Prgunta nro14:
Se depende indirectamente del area comercial para poder justificar la inversion realizada.
\item Prgunta nro15:
Si, considero que existen varios problemas, entre los cuales se pueden mencionar los siguientes:
\begin{itemize}
\item Por escasez de recursos el nivel de respuesta no es el adecuado.
\item Los proyectos de inversion tecnologica estan frenados.
\item No se esta cumpliendo el presupuesto estimado para el area.
\end{itemize}
En este momento de la empresa no esta en mis manos poder cambiarlo.
El nivel de recursos es muy bajo y realmente el presupuesto esta totalmente vedado.
\item Pregunta nro16:
Por lo mencionado anteriormente no exite evualuacion de desempeno.

\item Pregunta nro17:
En la parte de sistemas no se gestionan bonos.

\item Pregunta nro18:
No, no lo considero.
Los detalles del porque son los siguientes:
\begin{itemize}
\item La jefatura del area administrativa financiera esta sobredimensionada, es decir tienen 3 gerentes los cuales no son necesarios para la actualidad de la empresa.
\item No se cumple correctamente el proceso de facturacion y cobranza.
\item Es ineficiente el prodecimiento de conciliacion de cuentas.
\item No se hace el correcto analisis del mercado y su segmentacion.
\item La carga de datos es ineficiente.
\item No es correcto el enfoque de marketing de la empresa, es decir a donde se dirijen las publicidades.
\item Se recarga trabajo operativo y administrativo a los vendedores.
\item El area operativa funciona de forma desornada.
\item No estan bien definidas las responsabilidades y funciones de cada puesto de trabajo.
\item No existe un producto donde la empresa sea lider.
\item No cuenta la empresa con alguna certificacion, como por ejemplo podria ser la Q de calidad.
\item Procedimientos bastante informales.
\item Poca documentacion de los procedimientos de cada area que se podria estandarizar.
\item Disconformidad generalizada del recurso humano.
\end{itemize}
\item Pregunta nro19:
Por lo menos faltaria una persona para realizar de forma correcta y operativa las tareas del area, pero por reduccion de personal no se puede contar con la misma.

\section{Evolucion de la empresa en los ultimos tiempos}
Hasta el 2001 durante la convertibilidad la empresa manejaba un monoproducto que era la venta de vuelos charter con hoteleria al caribe.
Ante la salida de la convertibilidad se tuvo que cambiar el negocio dado que dejo de ser rentable, mas que nada ejecutable.
Al pasar esto la empresa se puso como objetivo ampliar sus operaciones a una empresa mayorista de tipo emisivo y receptivo.
Se abren los departamentos de ventas internacionales,nacionales y receptivo.
Evolucionaron positivamente hasta el principios del 2008 donde comenzo la crisis intitucional dentro del pais y comenzaron a sentirse los efectos de la crisis financiera mundial.
La crisis nacional afecto directamente en el negocio de venta de paquetes turisticos como de viajes dentro de la Argentina como hacia el exterior.
La internacional golpeo nuestro principal mercado para el receptivo, el espanol, lo cual hizo que el negocio receptivo se vea seriamente afectado.
La empresa entro en un proceso de reestructuracion reduciendo personal y recursos, pasando de 120 personas a alredor de 60.
La caida abrupta en las ventas genero un problema financiero llevando a la empresa a la necesidad de vender en algunos casos por debajo de los costos con la intencion de obtener liquidez. Para ello se potencio el negocio de venta de pasajes areo.
Algunos de los negocios que se trataron de incursionar pero no tuvieron exito fueron:
Desarrollar el mercado hacia China y la India.
En ambos destinos se invirtio dinero en marketing y publicidad no obteniendo casi ningun retorno.
En el proceso de reduccion ademas se pasaron de 2 plantas a 1 sola, reduciendo considerablemente el area de trabajo.
\end{itemize}
\section{Graciela Penas,Gerencia Administrativa}
\begin{itemize}
\item Prguntanro1:
Lic. En Administracion de Empresas y Contadora Publica.
Si, se corresponden.
Si, sigo haciendo cursos y actualizandome.
\item Prguntanro2:
16 anios como gerenta administrativa y financiera en un Molino Harinero( Molino Osiris)
Julia Tours, de forma independiente al igual que Marsans formando el departamento administrativo,finalmente llegue a Marsans.
\item Prguntanro3:
 Gerenta Administrativa
\item Prguntanro4:
Son las relacionadas con supervision de las registraciones contables de la empresa, conciliaciones de cuentas patrimoniales y de resultado, presentacion de informes mensuales y trimestrales, confeccion de balance mensual semestral y anual, pagos y presentaciones de impuestos y previsionales ante los organismos de control.
\item Prguntanro5:
El gerente General.
\item Prguntanro6:
Al gerente general o a directores internacionales.
\item Prguntanro7:
Actualmente por la reestructuracion de personal cuento con 2 personas a cargo a las que se le delega: Contabilizacion de facutra de proveedores, conciliaciones bancarias, conciliaciones contables y liquidacion de impuestos.
\item Prguntanro8:
Si, sobre el area de finanzas, por ejemplo pagos a proveedores, es decir, la conciliacion con los mismos e idem con cobranzas pero para la parte de deudores.
\item Prguntanro8:
No, son siempre nuevos.
\item Prguntanro9:
Las decisiones son tomadas en base a la informacion que se saca del sistema contable.
\item Prguntanro10:
Mensualmente al exterior los resultados del mes, ajuste de presupuesto o a las partidas presupuestarias y balances.
Internamente al director general la misma informacion.
\item Prguntanro11:
No se depende de ningun area.
\item Prguntanro12:
Si actualmente considero que es escaso el personal para realizar las tareas pertinentes del area.
Dada la situacion de la empresa no cuento con la posibilidad de cambio.
\item Prguntanro13:
No, no existe evaluacio.
\item Preguntaron14:
No, no existe. Hasta Enero.
\item Prguntanro15:
Considero que las las areas no se encuentran proporcionadas en cuanto a la cantidad de personal.
\item Prguntanro16:
Le falta personal, como mencione anteriormente.

\section{Comentarios Adicionales}
La empresa si bien es internacional, esta gerenciada como una pyme familiar.
La persona a cargo de la empresa no cuenta con ninguna formacion profesional, no sabiendo distinguir lo que es un activo de un pasivo,devengado de percibido,ni patrimonial de financiero haciendo muy dificil la interpretacion de un informe profesional interno o de los auditores externos tomando decisiones bajo caprichos.
No hay politicas de empresa, no hay planeamiento, no hay una integracion vertical. La comunicacion es de forma informal y no se utilizan los canones establecidos para ello.

\section{Mariana Rius,Gerencia Receptivo}
\begin{itemize}
\item Pregunta nro1:
Tecnica en turismo.
Si, corresponde para las competencias del cargo.
Actualmente no realizado estudios ni cursos.
\item Pregunta nro2:
Venta en empresa de viajes estudiantiles.
Ventas en distintas agencias de turismo(Rapido Argentino, Turismo la Frontera, Canarias Travel)
Aeropuerto de Ezeiz, contacto con los pasajeros
Finalmente llegue a Marsans.
\item Pregunta nro3:
Comercializacion del departamento de Receptivo y su parte operativa.
\item Pregunta nro4:
El Gerente General.
\item Pregunta nro5:
Al gerente general y al director.
\item Pregunta nro6:
Actualmente cuento con 4 subordinados y las tareas que delego sobre ellos son:
tareas operativas y comerciales siempre referido a receptivo.
\item Pregunta nro7:
Si, sobre la parte de administracion y documentacion.
\item Pregunta nro8:
Por lo general son problemas nuevos, si tiene una solucion se resulve de forma informal.
\item Pregunta nro9:
En base de lo que llega de los clientes y del estudio de mercado receptivo.
\item Pregunta nro10:
Ventas diarias, mensuales, anuales. Proyecciones a corto, largo y mediano plazo.
\item Pregunta nro11:
Dependemos de producto y del area de administracion.
De producto por que arman los productos para vender y lo cargan en el sistemas.
De administracion dependen los pagos de los clientes y los pagos a proveedores.
\item Pregunta nro12:
Idem que todos.
\item Pregunta nro13:
No existe evaluacion de la labor de los empleados.
\item Pregunta nro14:
No, no existe.
\item Pregunta nro15:
Considero que hay ciertas gerencias a las cuales les sobra personal jerarquico.
Hay ciertas disfunciones entre administracion y el resto de las areas.
\item Pregunta nro16:
Considerando el recurso humano es decir cantidad de horas que deberian trabajar las personas, hubiese un sistema de acorde, es decir actualizado y unos buenos procedimientos me considaria con el personal de acorde, pero en este momento se necesita una persona mas.
\end{itemize}

\section{Eugenio Kakias,Ventas no mando medio}
\begin{itemize}
\item Pregunta nro1:
Terciario y Universitarios incompletos, Medicina y Turismo.
Si, se corresponden con las competancias.
Si, mantengo los estudios actualizados.
\item Pregunta nro2:
Realice trabajos de atencion al cliente y venta en distintas agencias de turismo.
\item Pregunta nro3:
Realizo operaciones de venta.
\item Pregunta nro4:
Lo mismo que antes
\item Pregunta nro5:
mis superior directo es el de ventas, Daniel Zuazo.
\item Pregunta nro6:
Solamente a mi superior directo.
\item Pregunta nro7:
Lamentablemente no.
\item Pregunta nro8:
No.
\item Pregunta nro9:
Si, los mismos. Para resolverlos utilizo mecanismos que estan provistos por la empresa, en caso de no solucionarlos asi hay improvisacion.
\item Pregunta nro10:
La mas real posible, sacada principalmente de los sistemas de la compania y de internet.
\item Pregunta nro11:
La unica informacion que reporto son los problemas que no se me tiene permitido solucionar por mi cuenta.
\item Pregunta nro12:
Trato de ser coordial, por medio de mails y personalmente.
\item Pregunta nro13:
Dependemos en general de todas las areas.
\item Pregunta nro14:
El problema principal de la empresa es la direccion. No cuento con la posibilidad.
\item Pregunta nro15:
No
\item Pregunta nro16:
No
\item Pregunta nro17:
No corresponde al cargo

\end{itemize}

\section{Varone Nicolas,Gerente de Producto}
\item Pregunta nro1:
Lic. En Turismo y Master en Hoteleria.
Si, se corresponde.
Mantengo actualizado por el trabajo mismo, no se necesita especialiciones en general.
\item Pregunta nro2:
Receptivo como primer trabajo.
Luego varias areas de un hotel, desde la recepcion hasta ventas.
Luego en Marsans de Data Entry hasta el cargo actual
\item Pregunta nro3:
Contratacion de proveedores de la empresa, companias areas, hoteles, etc.
\item Pregunta nro4:
Negociacion de la empresa respecto a las funciones y el armado final del producto para ofrecer al mercado.
\item Pregunta nro5:
Director General
\item Pregunta nro6:
Director General y cada tanto a la gerencia comercial cuando la habia.
\item Pregunta nro7:
Mas o menos 10 personas y Data entry, operativo de producto, los que arman el producto,.
\item Pregunta nro8:
No, siempre son nuevos. Se resuelven con sentido comun
\item Pregunta nro9:
En base al sentido comun y a la experiencia
\item Pregunta nro10:
Todos los acuerdos comerciales con los proveedores y todo nuevo producto que se genera.
\item Pregunta nro11:
Anteriormente habia reuniones semanales de mandos donde se cruzaba informaciond de los departamentos. Actualmente casi no hay reuniones y la comunicacion es informal via correo.
\item Pregunta nro12:
No se depende directamente de un area.
\item Pregunta nro13:
El principal problema de la empresa no habia una integracion vertical, dado que no habia un lineamiento comercial, es decir era por intuicion de lo que se debia.
Dada la falta del lineamiento en la gerencia propia nunca me pude suplir desde mi lugar esa falta.
\item Pregunta nro14:
Dentro del departamento anteriormente se realizaba un segumiento del personal diario y con eso se evaluaba asi al mismo personal para eventuales decisiones.
No habia feedback de esto hacia el empleado.
\item Pregunta nro15:
No.
\item Pregunta nro16:
No, no lo considero.
En primer lugar la falta de capacidad.
En segundo lugar la falta de responsabilidad.
En tercer lugar la falta del recurso humano.
Hay demasiados departamentos para la estructura de la empresa.
Cada gerencia es como una pyme familiar en cuanto a que cada uno cuida lo suyo.
\item Pregunta nro17:
No.










\end{itemize}

\section{Confeccion\'o}
Fern\'andez Nicol\'as
\end{document}