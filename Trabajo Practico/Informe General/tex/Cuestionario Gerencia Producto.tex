\documentclass[12pt,a4paper,spanish]{article}
\usepackage[spanish]{babel}
\usepackage [T1]{fontenc}
\usepackage [latin1]{inputenc}
\usepackage{graphicx}
\usepackage{float}
\usepackage{array}
	  \oddsidemargin 0in
      \textwidth 6.75in
      \topmargin 0in
      \textheight 10.0in
      \parindent 0em
      \parskip 2ex
\usepackage{anysize}
\marginsize{3cm}{2cm}{1.0cm}{1.0cm}

\pagestyle{plain}

\begin{document}
\title{
\begin{table}[!h]
	\begin{tabular}{m{2cm}m{15cm}}
		\multicolumn{1}{l}{}
		 \includegraphics[scale=0.25, bb=0 0 0 0]{logo-fiuba.png} & 
		 \begin{center}
		 	\begin{LARGE}
				Universidad de Buenos Aires	\linebreak \linebreak		 							Facultad de Ingenier\'{i}a  \linebreak \linebreak
				7112 - Estrucutra de las Organizaciones \linebreak \linebreak
				2do. Cuatrimestre de 2009
			\end{LARGE}
		 \end{center}\\
\end{tabular}
\end{table}
\begin{Large}
 \begin{center}
		\underline{Entrevistas} \linebreak \linebreak
        Grupo Nro:	R2
\end{center}
\end{Large}
}
\date{}
\maketitle

\thispagestyle{empty}

\author{
\begin{Large}
\begin{center}
		\underline{Integrantes}  \linebreak 
\end{center}
\end{Large}
\begin{center}
	\begin{tabular}{|| c | c | c ||}
		\hline
		\begin{large}Apellido,Nombre\end{large} & 
		\begin{large}Padr\'{o}n Nro.\end{large} & 
		\begin{large}E-mail\end{large}\\
		\hline
		Bruno Tom\'as & 88.449 & tbruno88@gmail.com\\
		\hline
		Chiabrando Alejandra Cecilia & 86.863 & achiabrando@gmail.com\\
		\hline
		Fern\'{a}ndez Nicol\'{a}s  & 88.599 & nflabo@gmail.com\\
		\hline
		Invernizzi Esteban Ignacio & 88.817 & invernizzie@gmail.com\\
		\hline
		Medbo Vegard & \- & vegard.medbo@gmail.com\\
		\hline
		Meller Gustavo Ariel & 88.435 & gustavo\_meller@hotmail.com\\
		\hline
		Mouso Nicol\'as & 88.528 & nicolasgnr@gmail.com\\
		\hline
		Mu\~noz Facorro Juan Mart\'in & 84.672 & juan.facorro@gmail.com\\
		\hline
		Wolfsdorf Diego & 88.162 & diegow88@gmail.com\\
		\hline
	\end{tabular}
\end{center}
}

\newpage
\setcounter{page}{1}
\tableofcontents

\newpage
	\section{Gerencia de Producto:Varone Nicolas}

	\paragraph{Pregunta}
	 ?`Cu\'al es su formaci\'on?  ?`Corresponde a las competencias requeridas por el cargo?  ?`Mantiene actualizados sus conociemientos?
	\paragraph{Respuesta}
Me recibi de Licicenciado En Turismo y tengo un Master en Hoteleria.
Mi cargo se corresponde con las competencias del cargo.
Me mantengo actualizado por el trabajo mismo, no se necesita especialiciones en general.

	\paragraph{Pregunta}
	 ?`Cu\'al es su experiencia laboral?
	\paragraph{Respuesta}
Realice tareas como receptivo en mi primer trabajo en turismo.
Luego trabaje en varias areas de un hotel, desde la recepcion hasta ventas.
Finalemente llegue a Marsans como Data Entry hasta el cargo actual.

	\paragraph{Pregunta}
	 ?`Cu\'al es su funci\'on en la empresa?
	\paragraph{Respuesta}
La Contratacion de proveedores de la empresa, companias aereas, hoteles, etc.
	
	\paragraph{Pregunta}
	 ?`Cu\'al es su funci\'on?  ?`Cu\'ales son sus responsabilidades?
	\paragraph{Respuesta}
La Negociacion de la empresa respecto a las funciones y el armado final del producto para ofrecer al mercado.

	\paragraph{Pregunta}
	 ?`Qui\'enes son sus superiores? 
	\paragraph{Respuesta}
El Director General.

	\paragraph{Pregunta}
	 ?`A qui\'en le reporta?
	\paragraph{Respuesta}
Al Director General y cada tanto a la gerencia comercial cuando la habia.

	\paragraph{Pregunta}
	 ?`Qui\'enes son sus subordinados?  ?`Qu\'e tareas delega sobre ellos?
	\paragraph{Respuesta}
Mas o menos 10 personas. Las tareas que se delegan son de Data entry, operaciones de producto,es decir los que arman el producto.

	\paragraph{Pregunta}
	 ?`Delega tareas sobre personas de otras \'areas?
	\paragraph{Respuesta}
No.
	\paragraph{Pregunta}
	 ?`Se le presentan los mismos problemas a diario o con cierta periodicidad?  ?`Cu\'ales son y c\'omo los resuelve?
	\paragraph{Respuesta}
No, siempre son nuevos. Se resuelven con sentido comun

	\paragraph{Pregunta}
	 ?`En base a tipo de informaci\'on realiza sus decisiones?  ?`De d\'onde la obtiene?
	\paragraph{Respuesta}
	En base al sentido comun y a la experiencia. La informacion se saca del mercado y de los proveedores.

	\paragraph{Pregunta}
	 ?`Qu\'e informaci\'on debe reportar?
	\paragraph{Respuesta}
Todos los acuerdos comerciales con los proveedores y todo nuevo producto que se genera.

	\paragraph{Pregunta}
	 ?`Cu\'al es su relaci\'on y c\'omo se comunica con los dem\'as gerentes?
	\paragraph{Respuesta}
Anteriormente habia reuniones semanales de mandos donde se cruzaba informaciond de los departamentos. Actualmente casi no hay reuniones y la comunicacion es informal via correo.

	\paragraph{Pregunta}
	 ?`De la producci\'on de qu\'e \'areas depende la concreci\'on de la propia? (Interdependencia operativa)
	\paragraph{Respuesta}
No se depende de un area en particular.
	\paragraph{Pregunta}
	 ?`Considera que existe alg\'un problema con la estructura de la empresa o la organizaci\'on actual de su gerencia?  ?`Tiene la posibilidad de cambiarla?  ?`Planea hacerlo?
	\paragraph{Respuesta}
El principal problema de la empresa es que nunca hubo una integracion vertical, dado que no habia un lineamiento comercial, es decir era por intuicion de lo que se debia hacer.
Dada la falta del lineamiento en la gerencia propia, nunca me pude suplir desde mi lugar esa falta.

	\paragraph{Pregunta}
	 ?`Existe alg\'un tipo de evaluaci\'on de desempe\~{n}o?  ?`Cada cu\'anto tiempo se hace?  ?`Qu\'e aspectos se eval\'uan?  ?`Se encuentra conforme con su \'ultima evaluaci\'on recibida?
	\paragraph{Respuesta}
Dentro del departamento anteriormente se realizaba un segumiento del personal diario y con eso se evaluaba asi al mismo personal para eventuales decisiones.
No habia feedback de esto hacia el empleado.

	\paragraph{Pregunta}
	 ?`Existe alg\'un tipo sistema de bonos?  ?`C\'omo funciona?  ?`C\'omo se determina lo que le corresponde a cada empleado?
	\paragraph{Respuesta}
No.
	\paragraph{Pregunta}
	 ?`Considera adecuada la estructura de las dem\'as gerencias?  ?`Observa disfunciones?
	\paragraph{Respuesta}
No, no lo considero.
En primer lugar la falta de capacidad.
En segundo lugar la falta de responsabilidad.
En tercer lugar la falta del recurso humano.
Hay demasiados departamentos para la estructura de la empresa.
Cada gerencia es como una pyme familiar en cuanto a que cada uno cuida lo suyo.

	\paragraph{Pregunta}
	 ?`Considera que a su sector le sobra o le falta personal para realizar sus tareas?
	\paragraph{Respuesta}
	No, no lo considero
\end{document}