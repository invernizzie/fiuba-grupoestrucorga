\documentclass[12pt,a4paper,spanish]{article}
\usepackage[spanish]{babel}
\usepackage [T1]{fontenc}
\usepackage [latin1]{inputenc}
\usepackage{graphicx}
\usepackage{float}
\usepackage{array}
	  \oddsidemargin 0in
      \textwidth 6.75in
      \topmargin 0in
      \textheight 10.0in
      \parindent 0em
      \parskip 2ex
\usepackage{anysize}
\marginsize{3cm}{2cm}{1.0cm}{1.0cm}

\pagestyle{plain}

\begin{document}
\title{
\begin{table}[!h]
	\begin{tabular}{m{2cm}m{15cm}}
		\multicolumn{1}{l}{}
		 \includegraphics[scale=0.25, bb=0 0 0 0]{logo-fiuba.png} & 
		 \begin{center}
		 	\begin{LARGE}
				Universidad de Buenos Aires	\linebreak \linebreak		 							Facultad de Ingenier\'{i}a  \linebreak \linebreak
				7112 - Estrucutra de las Organizaciones \linebreak \linebreak
				2do. Cuatrimestre de 2009
			\end{LARGE}
		 \end{center}\\
\end{tabular}
\end{table}
\begin{Large}
 \begin{center}
		\underline{Entrevistas} \linebreak \linebreak
        Grupo Nro:	R2
\end{center}
\end{Large}
}
\date{}
\maketitle

\thispagestyle{empty}

\author{
\begin{Large}
\begin{center}
		\underline{Integrantes}  \linebreak 
\end{center}
\end{Large}
\begin{center}
	\begin{tabular}{|| c | c | c ||}
		\hline
		\begin{large}Apellido,Nombre\end{large} & 
		\begin{large}Padr\'{o}n Nro.\end{large} & 
		\begin{large}E-mail\end{large}\\
		\hline
		Bruno Tom\'as & 88.449 & tbruno88@gmail.com\\
		\hline
		Chiabrando Alejandra Cecilia & 86.863 & achiabrando@gmail.com\\
		\hline
		Fern\'{a}ndez Nicol\'{a}s  & 88.599 & nflabo@gmail.com\\
		\hline
		Invernizzi Esteban Ignacio & 88.817 & invernizzie@gmail.com\\
		\hline
		Medbo Vegard & \- & vegard.medbo@gmail.com\\
		\hline
		Meller Gustavo Ariel & 88.435 & gustavo\_meller@hotmail.com\\
		\hline
		Mouso Nicol\'as & 88.528 & nicolasgnr@gmail.com\\
		\hline
		Mu\~noz Facorro Juan Mart\'in & 84.672 & juan.facorro@gmail.com\\
		\hline
		Wolfsdorf Diego & 88.162 & diegow88@gmail.com\\
		\hline
	\end{tabular}
\end{center}
}

\newpage
\setcounter{page}{1}
\tableofcontents

\newpage
	\section{Gerencia Administrativa:Graciela Penas}

	\paragraph{Pregunta}
	 ?`Cu\'al es su formaci\'on?  ?`Corresponde a las competencias requeridas por el cargo?  ?`Mantiene actualizados sus conociemientos?
	\paragraph{Respuesta}
Me recibi de Licenciada en Administracion de Empresas y Contadora Publica.
Mi cargo excede las competencias requeridas actualmente por la empresa.
Sigo haciendo cursos y actualizandome.

	\paragraph{Pregunta}
	 ?`Cu\'al es su experiencia laboral?
	\paragraph{Respuesta}
16 anios como gerenta administrativa y financiera en un Molino Harinero( Molino Osiris)
Julia Tours, de forma independiente al igual que Marsans formando el departamento administrativo,finalmente llegue a Marsans de forma efectiva.

	\paragraph{Pregunta}
	 ?`Cu\'al es su funci\'on en la empresa?
	\paragraph{Respuesta}
	 Gerenta Administrativa

	\paragraph{Pregunta}
	 ?`Cu\'al es su funci\'on?  ?`Cu\'ales son sus responsabilidades?
	\paragraph{Respuesta}
	Son las relacionadas con supervision de las registraciones contables de la empresa, conciliaciones de cuentas patrimoniales y de resultado, presentacion de informes mensuales y trimestrales, confeccion de balance mensual semestral y anual, pagos y presentaciones de impuestos y previsionales ante los organismos de control.

	\paragraph{Pregunta}
	 ?`Qui\'enes son sus superiores? 
	\paragraph{Respuesta}
El gerente General.

	\paragraph{Pregunta}
	 ?`A qui\'en le reporta?
	\paragraph{Respuesta}
Al gerente general o a directores internacionales.

	\paragraph{Pregunta}
	 ?`Qui\'enes son sus subordinados?  ?`Qu\'e tareas delega sobre ellos?
	\paragraph{Respuesta}
Actualmente por la reestructuracion de personal cuento con 2 personas a cargo a las que se le delega: Contabilizacion de facutra de proveedores, conciliaciones bancarias, conciliaciones contables y liquidacion de impuestos.

	\paragraph{Pregunta}
	 ?`Delega tareas sobre personas de otras \'areas?
	\paragraph{Respuesta}
Si, sobre el area de finanzas, por ejemplo pagos a proveedores, es decir, la conciliacion con los mismos e idem con cobranzas pero para la parte de deudores.

	\paragraph{Pregunta}
	 ?`Se le presentan los mismos problemas a diario o con cierta periodicidad?  ?`Cu\'ales son y c\'omo los resuelve?
	\paragraph{Respuesta}
No, son siempre nuevos.

	\paragraph{Pregunta}
	 ?`En base a tipo de informaci\'on realiza sus decisiones?  ?`De d\'onde la obtiene?
	\paragraph{Respuesta}
Las decisiones son tomadas en base a la informacion que se saca del sistema contable.
	
	\paragraph{Pregunta}
	 ?`Qu\'e informaci\'on debe reportar?
	\paragraph{Respuesta}
Mensualmente al exterior los resultados del mes, ajuste de presupuesto o a las partidas presupuestarias y balances.
Internamente al director general la misma informacion.

	\paragraph{Pregunta}
	 ?`Cu\'al es su relaci\'on y c\'omo se comunica con los dem\'as gerentes?
	\paragraph{Respuesta}
Se realiza mediante los lineamientos prefijados, es decir, en el caso de esta area se respetan los procedimientos.

	\paragraph{Pregunta}
	 ?`De la producci\'on de qu\'e \'areas depende la concreci\'on de la propia? (Interdependencia operativa)
	\paragraph{Respuesta}
No se depende de ningun area.

	\paragraph{Pregunta}
	 ?`Considera que existe alg\'un problema con la estructura de la empresa o la organizaci\'on actual de su gerencia?  ?`Tiene la posibilidad de cambiarla?  ?`Planea hacerlo?
	\paragraph{Respuesta}
Si actualmente considero que es escaso el personal para realizar las tareas pertinentes del area.
Dada la situacion de la empresa no cuento con la posibilidad de cambio.

	\paragraph{Pregunta}
	 ?`Existe alg\'un tipo de evaluaci\'on de desempe\~{n}o?  ?`Cada cu\'anto tiempo se hace?  ?`Qu\'e aspectos se eval\'uan?  ?`Se encuentra conforme con su \'ultima evaluaci\'on recibida?
	\paragraph{Respuesta}
No, no existe evaluacion.

	\paragraph{Pregunta}
	 ?`Existe alg\'un tipo sistema de bonos?  ?`C\'omo funciona?  ?`C\'omo se determina lo que le corresponde a cada empleado?
	\paragraph{Respuesta}
No, no existe. Hasta Enero exitia pero se suprimio por la actualidad de la empresa.

	\paragraph{Pregunta}
	 ?`Considera adecuada la estructura de las dem\'as gerencias?  ?`Observa disfunciones?
	\paragraph{Respuesta}
Considero que las las areas no se encuentran proporcionadas en cuanto a la cantidad de personal.

	\paragraph{Pregunta}
	 ?`Considera que a su sector le sobra o le falta personal para realizar sus tareas?
	\paragraph{Respuesta}
	Le falta personal, como mencione anteriormente.

\subsection{Comentarios Adicionales}
La empresa si bien es internacional, esta gerenciada como una pyme familiar.
La persona a cargo de la empresa no cuenta con ninguna formacion profesional, no sabiendo distinguir lo que es un activo de un pasivo,devengado de percibido,ni patrimonial de financiero haciendo muy dificil la interpretacion de un informe profesional interno o de los auditores externos tomando decisiones bajo caprichos.
No hay politicas de empresa, no hay planeamiento, no hay una integracion vertical. La comunicacion es de forma informal y no se utilizan los canones establecidos para ello.


\end{document}