\documentclass[12pt,a4paper,spanish]{article} 
\usepackage{babel}
\usepackage [T1]{fontenc}
\usepackage [latin1]{inputenc}
\usepackage{graphicx}
\usepackage{array}
	  \oddsidemargin 0in
      \textwidth 6.75in
      \topmargin 0in
      \textheight 10.0in
      \parindent 0em
      \parskip 2ex
\usepackage{anysize}
\marginsize{3cm}{2cm}{1.0cm}{1.0cm}

\pagestyle{plain}

\begin{document} 
\title{
\begin{table}[!h]
	\begin{tabular}{m{2cm}m{15cm}}
		\multicolumn{1}{l}{}
		 \includegraphics[scale=0.25, bb=0 0 0 0]{Logo-fiuba.PNG} & 
		 \begin{center}
		 	\begin{LARGE}
				Universidad de Buenos Aires	\linebreak \linebreak		 							Facultad de Ingenier\'{i}a  \linebreak \linebreak
				7112 - Estrucutra de las Organizaciones \linebreak \linebreak
				2do. Cuatrimestre de 2009
			\end{LARGE}
		 \end{center}\\
\end{tabular}
\end{table}
\begin{Large}
 \begin{center}
		\underline{Trabajo Pr\'{a}tico} \linebreak \linebreak
        Grupo Nro:	R2
\end{center}
\end{Large}
}
\date{}
\maketitle

\thispagestyle{empty}
\author{
\begin{Large}
\begin{center}
		\underline{Integrantes}  \linebreak 
\end{center}
\end{Large}
\begin{center}
	\begin{tabular}{|| c | c | c ||}
		\hline
		\begin{large}Apellido,Nombre\end{large} & 
		\begin{large}Padr\'{o}n Nro.\end{large} & 
		\begin{large}E-mail\end{large}\\
		\hline
		Bruno Tom\'as & 88449 & tbruno88@gmail.com\\
		\hline
		Chiabrando Alejandra Cecilia & 86.863 & achiabrando@gmail.com\\
		\hline
		Fern\'{a}ndez Nicol\'{a}s  & 88.599 & nflabo@gmail.com\\
		\hline
		Invernizzi Esteban Ignacio & 88.817 & invernizzie@gmail.com\\
		\hline
		Medbo Vegard & \- & vegard.medbo@gmail.com\\
		\hline
		Meller Gustavo Ariel & 88.435 & gustavo\_meller@hotmail.com\\
		\hline
		Mouso Nicol\'as & 88.528 & nicolasgnr@gmail.com\\
		\hline
		Mu\~noz Facorro Juan Mart\'in & 84.672 & juan.facorro@gmail.com\\
		\hline
		Wolfsdorf Diego & 88.162 & diegow88@gmail.com\\
		\hline
	\end{tabular}
\end{center}
}

\newpage
\setcounter{page}{1} 
\tableofcontents
\newpage
\section{Alcance del Trabajo Pr\'{a}ctico}

\section{Comparaci\'{o}n de Empresas}

\subsection{Marsans}
\subsubsection{Disponibilidad}
Nicol\'{a}s Fern\'andez, uno de los integrantes de nuestro grupo trabaj\'{o} en la empresa alrededor de dos a\~nos. Eso hizo que forme una buena relaci\'{o}n con algunos de los integrantes de la misma. Hablando con uno de los gerentes le coment\'{o} la idea de trabajar con Marsans Argentina S.A. a lo que el gerente le comunic\'{o} que no habr\'{i}a ning\'{u}n problema, que simplemente avisando con un peque\~no tiempo de anticipaci\'on, de no m\'{a}s de 2 o 3 d\'{i}as, podr\'{i}amos concretar las entrevistas o recavar la informaci\'on necesaria por medios electr\'onicos. Esto nos di\'{o} una pauta de que la disponibilidad es muy buena y que la empresa tiene la predisposici\'{o}n que necesitamos, aunque con una peque\~na demora, por lo cual el puntaje elegido fue un 8.

\subsubsection{Contacto}
Nicol\'{a}s Fern\'andez tiene contacto directo con: Sonia Kraimer (Gerencia de Recursos Humanos) y Alejandro Zubiria (Supervisor de Sistemas), con los cuales mantiene una muy buena relaci\'{o}n. A su vez conoce a distintas personas de otras \'{a}reas y tambi\'{e}n a trav\'{e}s de estas dos personas mencionadas puede obtener el contacto con cualquier otra persona que se necesite, en cualquier nivel jer\'arquico. Los contactos de la empresa nos parecieron muy predispuestos y creemos que vamos a poder acceder a cualquier contacto de otra \'{a}rea con lo cual el puntaje elegido fue un 10.

\subsubsection{Estructura}
La empresa es reconocida dentro del pa\'{i}s. La misma cuenta con un organigrama formal. Tiene diversos departamentos, sectores y puestos en los que se delegan responsabilididas y disminuye progresivamente la autoridad formal. Esto resulta una estructura ideal para el proyecto que realizaremos. El puntaje elegido fue un 10.

\subsubsection{Conocimiento de la Estructura}
Los contactos que tenemos en la empresa tienen amplio conococimiento acerca de la estructura, al igual que Nicol\'{a}s Fern\'andez, quien trabaj\'{o} all\'{i} en el \'area de Sistemas, que asiste al funcionamiento de todas las dem\'as, con lo cual se tiene un conocimiento bastante vasto sobre la misma y el hecho de que un integrante del grupo la conozca nos resulta muy \'{u}til. El puntaje elegido fue un 9.

\subsubsection{Ubicaci\'{o}n}
La empresa se encuentra en Suipacha y Avenida Santa F\'{e}, muy cerca del centro de la ciudad con lo cual se puede llegar cas\'{i} desde cualquier punto y queda relativamente cerca de ambas sedes de la facultad, tanto Las Heras como Paseo Col\'{o}n. A pesar de que no todos los integrantes vivimos en la Ciudad de Bs. As., estamos de acuerdo en que la ubicaci\'{o}n es excelente al estar en plena ciudad y tener muy f\'{a}cil acceso y abundantes medios de transporte desde y hacia cualquier ubicaci\'on. El puntaje elegido fue un 10.

\subsubsection{Tama\~{n}o}
La empresa cuenta con alrededor de 70 empleados, con lo cual el n\'umero se encuentra dentro del rango \'optimo sugerido por la c\'atedra y nos parece un buen n\'{u}mero al no ser muy grande ni muy chico. El puntaje elegido fue un 10.

\subsection{Software Factory, SAP y Consultor\'{i}a}
\subsubsection{Disponibilidad}

Uno de los integrantes del grupo que presenta este informe trabaja actualmente en la empresa referida. A pesar de no ser una empresa que supere los 200 empleados, la actitud que se percibe de la gerencia es altamente burocr\'{a}ctica. Es por esta raz\'{o}n que a pesar de tener una buena relaci\'{o}n con el gerente del \'{a}rea de Software Factory, la posibilidad de obtener entrevistas o informaci\'{o}n sobre la empresa se ve sujeta a la buena voluntad de las otras partes involucradas, lo cual es poco menos que una seguridad, si lo que se quiere es una alta disponibilidad. Es por esto que el puntaje asignado a este aspecto es 7.

\subsubsection{Contacto}

El contacto principal, como se mencion\'{o} anteriormente, es el gerente del \'{a}rea de Software Factory, con el cual se tiene una muy buena relaci\'{o}n laboral y personal. Se podr\'{i}a recurrir tambi\'{e}n a la ayuda de contactos en el \'{a}rea de recursos humanos y finanzas para la obtenci\'{o}n de informaci\'{o}n, aunque ya no en el nivel gerencial, lo cual agilizar\'{i}a el proceso de consulta pero limitar\'{i}a el acceso al tipo de datos que se pueden conseguir. El puntaje asignado a este aspecto es 8.

\subsubsection{Estructura}

Luego de la compra de una empresa dedicada al servicio de consultor\'{i}a SAP y la venta de un data center, se produjo una importante reforma en la estructura de la empresa, la cual incluyo recortes de personal, tanto a nivel gerencial, como administrativo y operativo. El nuevo organigrama fue dado a conocer a todos los empleados, por lo cual \'{e}ste se encuentra disponible para los objetivos de este informe. El puntaje asignado a este aspecto es 9.

\subsubsection{Conocimiento de la Estructura}

A pesar de estar disponible el nuevo organigrama, el conocimiento de la estructura real de la empresa no es muy profundo, dado que la re-estructuracion tuvo lugar recientemente por lo que el funcionamiento del \'{a}rea de SAP y las consultor\'{i}as relacionadas se ignoran. Esta l\'{i}nea de negocios es una de las m\'{a}s grandes que tiene actualmente la empresa, por lo tanto se desconoce la estructura de un porcentaje importante de la misma. El puntaje asignado a este aspecto es 6.

\subsubsection{Ubicaci\'{o}n}

Las oficinas se encuentran distribuidas entre un edificio en San Telmo, cerca de la Facultad de Ingenier\'{i}a, y otro en Mart\'{i}nez, zona norte del Gran Buenos Aires. En la sede de San Telmo funciona la Software Factory y el Call Center, mientras que en Mart\'{i}nez se desarrollan las actividades de SAP y consultor\'{i}a. El puntaje asignado a este aspecto es 10.

\subsubsection{Tama\~{n}o}

El n\'{u}mero de empleados, teniendo en cuenta todas las l\'{i}neas de negocios, se encuentra entre 150 y 200, con lo cual seg\'{u}n los par\'{a}metros establecidos por la c\'{a}tedra se encuentra dentro del n\'{u}mero deseado, aunque ser\'{i}a un poco grande si la estimaci\'{o}n obtenida es inexacta y la realidad supera el m\'{a}ximo considerado. El puntaje asignado a este aspecto es 8. 

\subsection{Constructora: Obras Ferroviarias}
\subsubsection{Disponibilidad}

Esta empresa parece estar en un constante ambiente de cambio y necesitar de un constante control y coordinaci\'{o}n de las tareas a realizar, por lo que el tiempo que pueden ofrecer los empleados es bastante reducido. El puntaje asignado a este aspecto es 6.

\subsubsection{Contacto}

La hermana de uno de los integrantes del grupo que presenta este informe trabaja en la empresa como secretaria tanto del gerente de Administraci\'{o}n y Finanzas, como del gerente de Producci\'{o}n. A pesar de ser este un contacto s\'{o}lido y seguro, la informaci\'{o}n quiz\'{a}s m\'{a}s sensible y espec\'{i}fica, queda fuera de su alcance por lo que habr\'{i}a que recurrir a la gerencia la cual, como se menciona en el punto anterior, dispone de muy baja disponibilidad. El puntaje asignado a este aspecto es 6.

\subsubsection{Estructura}

El organigrama de la empresa ha sido solicitado varias veces, pero el mismo nunca fue, al d\'{i}a de la fecha, confeccionado e informado a sus empleados. Parecer\'{i}a que el nivel de desorganizaci\'{o}n que se deduce de los comentarios del funcionamiento de la empresa, est\'{a} fuertemente ligado con el hecho de que \'{e}sta carece de una estructura formal que le permita mejorar su rendimiento. El puntaje asignado a este aspecto es 5.

\subsubsection{Conocimiento de la Estructura}

La informaci\'{o}n disponible sobre la estructura formal de la empresa es casi nula incluso para sus propios empleados, lo que marca una tendencia desfavorable a la hora de considerarla como sujeto de an\'{a}lisis. El puntaje asignado a este aspecto es 5.

\subsubsection{Ubicaci\'{o}n}

Las oficinas administrativas est\'{a}n localizadas en Retiro, en las que se puede encontrar a la gerencia de la empresa, por ende donde se concentra la mayor cantidad de informaci\'{o}n relevante para este informe. El puntaje asignado a este aspecto es 8. 

\subsubsection{Tama\~{n}o}

La cantidad de empleados asciende a m\'{a}s de 400, incluyendo, adem\'{a}s del personal administrativo, el operativo y el de obra. Debido a que este n\'{u}mero sobrepasa los l\'{i}mites sugeridos, el puntaje asignado a este aspecto es 4.

\subsection{Servicios e Intrumentaci\'{o}n para el Control Ambiental Industrial}
\subsubsection{Disponibilidad}

Debido a una auditor\'{i}a por la que debe pasar la empresa, el tiempo del que disponen los empleados es reducido, ya que de la aprobaci\'{o}n de esta auditor\'{i}a depende la retenci\'{o}n de su cliente m\'{a}s importante. A pesar de esto, la empresa estar\'{i}a dispuesta a brindar todo tipo de informaci\'{o}n. El puntaje asignado a este aspecto es 8.

\subsubsection{Contacto}

El hijo del due\~{n}o de la empresa, el cual trabaja en el \'{a}rea de Comercializaci\'{o}n de \'{e}sta, es un amigo muy cercano de unos de los integrantes del grupo que presenta este informe. Esta relaci\'{o}n asegura la buena voluntad por parte de la empresa de ayudar, en la medida que le sea posible, con cualquier tipo de informaci\'{o}n necesaria. El puntaje asignado a este aspecto es 8.

\subsubsection{Estructura}

Como resultado de la auditor\'{i}a a la que es sometida, se est\'{a}n realizando cambios estructurales y de procesos en la empresa, por lo cual, podr\'{i}a decirse que un an\'{a}lisis que se haga en esta situaci\'{o}n posiblemente resultar\'{i}a ambiguo y poco \'{u}til. El puntaje asignado a este aspecto es 6.

\subsubsection{Conocimiento de la Estructura}

La informaci\'{o}n que se tiene sobre la actual estructura de la empresa es obsoleta, debido al ya mencionado proceso de re-estructuraci\'{o}n al cual est\'{a} siendo sometida. El puntaje asignado a este aspecto es 5.

\subsubsection{Ubicaci\'{o}n}

Cuenta con oficinas en el barrio de Belgrano, donde se encuentran las oficinas centrales de la empresa, y otras en la provincia de Neuqu\'{e}n, las cuales no ser\'{i}an accesibles para los fines de este informe. El puntaje asignado a este aspecto es 7.

\subsubsection{Tama\~{n}o}

La empresa en su totalidad cuenta con 20-30 empleados, distribu\'{i}dos entre las dos oficinas mencionadas en el punto anterior. El puntaje asignado a este aspecto es 5.

\subsection{Tabla Comparativa}

\small
\begin{center}
\begin{tabular}{|| c | c | c | c | c | c | c ||}
\hline
\hline
T\'{e}rminos & Peso & Soft.Fact. & Marsans & Obras Ferr. & Sist.Ctrl.Adm. & Manuf. Avellaneda\\
\hline
Disponib.  & 10 & 7 & 8 & 6 & 8 & 8 \\
\hline
Contacto   & 10 & 8 & 10 & 6 & 10 & 10 \\
\hline
Estructura & 9 & 9 & 10 & 5 & 6 & 8 \\
\hline
Conoc. Estr. & 9 & 6 & 9 & 5 & 5 & 10 \\
\hline
Ubicaci\'{o}n & 8 & 10 & 10 & 8 & 7 & 5 \\
\hline
Tama\~{n}o   & 7 & 8 & 10 & 4 & 5 & 10 \\
\hline
\hline
Totales & - & 421 & 521 & 302 & 370 & 452 \\
\hline

\end{tabular}
\end{center}

\subsection{Justificaci\'{o}n de la Selecci\'{o}n de la Empresa}

Una vez hecha la tabla de comparaci\'{o}n con todos los items que cre\'imos importantes, la empresa que sum\'o el mayor puntaje fue Marsans Argentina S.A.
Al elegir los pesos de la tabla de comparaci\'{o}n, en lo que m\'as \'enfasis se puso fue tanto en la disponibilidad como el contacto, teniendo un segundo plano la estructura y el conocimiento de la estrucutra y siendo las caracter\'isticas con menor relevancia, la ubicaci\'on y el tama\~no.
Marsans obtuvo puntajes similares para casi todos los items a evaluar a diferencia de las dem\'{a}s empresas, en las cuales fueron muy variados los puntajes; algunas con puntajes muy altos para algunos items y muy bajos para otros y otras teniendo una regularidad de puntajes en todos los items, pero siendo estos bajos.
En s\'{i}ntesis Marsans Argentina S.A. nos pareci\'{o} la empresa que mejor se puede amoldar al trabajo que debemos realizar y creemos que vamos a poder tener toda la informaci\'{o}n que necesitemos de la misma cuando la requerramos.
A continuaci\'{o}n detallamos el porque de la elecci\'{o}n de los puntajes de cada item en particular, con una breve explicaci\'{o}n de cada uno.


En s\'{i}ntesis creemos que la empresa elegida es la mejor para lo que vamos a tener que trabajar. La misma cuenta con todo lo pedido y esperamos que la elecci\'{o}n de la misma confirme a lo largo del trabajo que es la opci\'on m\'as adecuada para el mismo, por todo lo planteado en este documento.

\section{Historia de la Empresa}
\section{Minutas de Reuni\'{o}n}
\section{Casos de An\'{a}lisis}
\subsection{Elavadores H\'{e}rcules}

\end{document}