
\documentclass[10pt]{article}
\pagestyle{plain}

\usepackage{tabulary}
\usepackage{graphicx}
\usepackage{float}
\usepackage[spanish]{babel}
\usepackage[ansinew]{inputenc}

\begin{document}
% Titulo principal del documento.
\title{		\textbf{Marsans Argentina} }

% Informacion sobre los autores.
\author{Grupo R2}

\date{}	   	   
% Inserta el titulo.
\maketitle
% Quita el numero en la primer pagina.

\thispagestyle{empty}

	Marsans Argentina es una empresa de venta de paquetes turisticos que forma parte del Grupo Marsans, con sucursales en varios paises de Europa y Am�rica.	

\subsubsection*{Problemas principales}
	\begin{enumerate}
		\item \emph{Estructura Organizacional:} Desproporcionada dimensi�n del estrato gerencial respecto al personal restante. Falta de confianza en la direcci�n de la compa�ia.
		\item \emph{Relacionados al Producto:} Se han tomado decisiones incorrectas relacionadas con la inversi�n en los paquetes armados para vender ya que no se han utilizado las investigaciones de mercado para el dise�o y desarrollo de productos.
		\item \emph{Marketing:} El nombre de Marsans ha sido desprestigiado en los �ltimos a�os debido a los problemas que tuvo durante el gerenciamiento de Aerol�neas Argentinas y su conflictivo desenlace.
		\item \emph{Problemas financieros:} Atraso en el pago de los sueldos y fuerte impacto de la crisis mundial.
	\end{enumerate}
	
	Es en esta situaci�n de crisis que el Grupo Marsans necesita de los servicios de una consultora para que analice la estructura de su filial en Argentina, con el f�n de poder llevar adelante un plan de salvataje.

\subsubsection*{Estrategia que se adoptar� para encarar los problemas}
	Reducir la estructura de la empresa para mejorar la comunicaci�n y lograr una mayor integraci�n entre los Departamentos, orient�ndolos de forma com�n hacia el objetivo de la organizaci�n, que es el de aportar valor a sus productos para obtener una ventaja competitiva. 
	Centrar a la empresa en pocos productos de buen rendimiento y demandados con escasas fluctuaciones, para lograr estabilidad financiera. 
	Partir de la base de que la motivaci�n de los empleados logra un mayor compromiso con el objetivo. 
	Redistribuir la fuerza de trabajo entre las distintas �reas, para evitar la p�rdida de personal con inducci�n y la contrataci�n de personal poco capacitado y/o sin experiencia, lo cual genera un costo adicional.


		
\end{document}
