\documentclass[12pt,a4paper,spanish]{article}
\usepackage[spanish]{babel}
\usepackage [T1]{fontenc}
\usepackage [latin1]{inputenc}
\usepackage{graphicx}
\usepackage{float}
\usepackage{array}
	  \oddsidemargin 0in
      \textwidth 6.75in
      \topmargin 0in
      \textheight 10.0in
      \parindent 0em
      \parskip 2ex
\usepackage{anysize}
\marginsize{3cm}{2cm}{1.0cm}{1.0cm}

\pagestyle{plain}

\begin{document}
\title{
\begin{table}[!h]
	\begin{tabular}{m{2cm}m{15cm}}
		\multicolumn{1}{l}{}
		 \includegraphics[scale=0.25, bb=0 0 0 0]{logo-fiuba.png} & 
		 \begin{center}
		 	\begin{LARGE}
				Universidad de Buenos Aires	\linebreak \linebreak		 							Facultad de Ingenier\'{i}a  \linebreak \linebreak
				7112 - Estrucutra de las Organizaciones \linebreak \linebreak
				2do. Cuatrimestre de 2009
			\end{LARGE}
		 \end{center}\\
\end{tabular}
\end{table}
\begin{Large}
 \begin{center}
		\underline{Entrevistas} \linebreak \linebreak
        Grupo Nro:	R2
\end{center}
\end{Large}
}
\date{}
\maketitle

\thispagestyle{empty}

\author{
\begin{Large}
\begin{center}
		\underline{Integrantes}  \linebreak 
\end{center}
\end{Large}
\begin{center}
	\begin{tabular}{|| c | c | c ||}
		\hline
		\begin{large}Apellido,Nombre\end{large} & 
		\begin{large}Padr\'{o}n Nro.\end{large} & 
		\begin{large}E-mail\end{large}\\
		\hline
		Bruno Tom\'as & 88.449 & tbruno88@gmail.com\\
		\hline
		Chiabrando Alejandra Cecilia & 86.863 & achiabrando@gmail.com\\
		\hline
		Fern\'{a}ndez Nicol\'{a}s  & 88.599 & nflabo@gmail.com\\
		\hline
		Invernizzi Esteban Ignacio & 88.817 & invernizzie@gmail.com\\
		\hline
		Medbo Vegard & \- & vegard.medbo@gmail.com\\
		\hline
		Meller Gustavo Ariel & 88.435 & gustavo\_meller@hotmail.com\\
		\hline
		Mouso Nicol\'as & 88.528 & nicolasgnr@gmail.com\\
		\hline
		Mu\~noz Facorro Juan Mart\'in & 84.672 & juan.facorro@gmail.com\\
		\hline
		Wolfsdorf Diego & 88.162 & diegow88@gmail.com\\
		\hline
	\end{tabular}
\end{center}
}

\newpage
\setcounter{page}{1}
\tableofcontents

\newpage
\section{An\'alisis de la situaci\'on de la empresa}

\subsection{Problemas en la estructura organizacional}

	Este problema se debe principalmente a la existencia de relaciones informales dentro de la organizaci\'on y la utilizaci\'on del ajuste mutuo como mecanismo coordinador. El ajuste mutuo es utilizado sin mayores problemas en organizaciones simples, en donde la cantidad de personal es peque�a, pero no plantea una soluci\'on adecuada para la relaci\'on entre el personal dentro de una empresa de mayor envergadura ya que complica la divisi\'on de tareas, las cuales pueden comenzar a superponerse, y tambi\'en la coordinaci\'on del trabajo.
	
\subsection{Problemas en el manejo de los departamentos y la comunicaci\'on entre los mismos}

	A partir de las entrevistas hechas, muchos de los gerentes coincidieron en que la empresa se maneja como una PYME familiar. Citando las respuestas de los gerentes: 
\begin{itemize}
	\item \emph{Cada gerencia es como una pyme familiar en cuanto a que cada uno cuida lo suyo.} - Departamento de Producto
	\item \emph{La empresa si bien es internacional, esta gerenciada como una pyme familiar.} - Departamento Administrativo
\end{itemize} 
	La coordinaci\'on de tareas deber\'ia ser llevaba a cabo respetando el organigrama y respetando los canales de comunicaci\'on establecidos. En un principio los gerentes se comunicaban entre s\'i mediante reuniones semanales en los cuales se discut\'ia la situaci\'on de cada departamento. Esto cambi\'o hasta el punto de que hoy en d\'ia, las conversaciones se desarrollan de una forma muy informal a trav\'es de e-mails o personalmente. Se cita:
	
\begin{itemize}
\item \emph{Trato de ser cordial, por medio de mails y personalmente.} - Vendedor
\item \emph{Anteriormente hab�a reuniones semanales de mandos donde se cruzaba informaci�n de los departamentos. Actualmente casi no hay reuniones y la comunicaci�n es informal v�a correo.} - Gerente del Departamento de Producto
\item \emph{Es dentro de todo fluida, pero informal.} - Gerencia receptiva.
\item \emph{La comunicaci�n es de forma informal y no se utilizan los canones establecidos para ello} - Gerente del Departamento de Administraci�n.
\item \emph{Procedimientos bastante informales.} - Gerente de Sistemas.
\end{itemize}

Esto se desarrolla en un marco en el cual a\'un se trata de mantener los lineamientos prefijados, respetando los procedimientos. Se puede concluir que la empresa necesita una reestructuraci�n interna para que la comunicaci�n se desarrolle de la forma que mejor se adapte a la situaci\'on actual.


Cada miembro de la empresa debe conocer tanto su funci\'on y responsabilidades, como que informaci\'on debe reportar y a quien debe hacerlo.
Seg�n del Gerente de Sistemas, no est\'an bien definidas las responsabilidades y funciones de cada puesto de trabajo. Tambi\'en resulta necesario que el personal de cada puesto sea el adecuado. Con la informaci�n obtenida, se puede concluir que hay gente en puestos cuyo cargo supera las capacidades de la persona. Seg\'un el Genrente del Departamento de Sistemas:
\begin{quotation}
Se recarga trabajo operativo y administrativo a los vendedores.
\end{quotation}
Cita de su par de Administraci�n:
\begin{quotation}
La persona a cargo de la empresa no cuenta con ninguna formaci\'on profesional, no sabiendo distinguir lo que es un activo de un pasivo, devengado de percibido, ni patrimonial de financiero haciendo muy dif�cil la interpretaci�n de un informe profesional interno o de los auditores externos tomando decisiones bajo caprichos. 
\end{quotation}

\subsection{Necesidad de reestructuraci\'on de la estructura departamental}
	Dada la situaci\'on actual de la empresa resulta necesario hacer una reestructuraci\'on de los departamentos existentes. El organigrama confeccionado, no representa la realidad de la empresa hoy en d\'ia. \'Este no ha sido modificado a pesar de las modificaciones que se fueron haciendo en la estructura de la organizaci\'on. Existen departamentos que no figuran en el organigrama, tal como es el caso del \'area de recepci\'on y otros, como el departamento de Compras, ya no existen y a\'un as\'i siguen figurando.
	
	Tambi\'en resulta necesaria una reducci\'on en el n\'umero de departamentos debido a los despidos sustanciales de personal. En palabras del Gerente de Producto: \emph{Hay demasiados departamentos para la estructura de la empresa}. En la mayor�a de los departamentos, la cantidad de empleados son 3 o 4 en promedio, lo cual hace innecesario que sean supervisados por un gerente. En algunos casos, resulta posible una fusi\'on de departamentos cuyas tareas est\'an \'intimamente relacionadas. En momentos en que se ha llevado a cabo una importante reducci\'on de personal, la empresa sigue manteniendo una estructura acorde una empresa de mayor tama\~{n}o. Una reducci\'on del n\'umero de departamentos podr\ia reducir los problemas de comunicaci\'on que se tienen en la actualidad.
	
	Proponemos entonces la fusi\'on del Departamento de Finanzas con el de Administraci\'on bas\'andonos en que el Departamento de Administraci\'on realiza muchas de las tareas que corresponden al Departamento de Finanzas. Cuando se le pregunta al Gerente de Administraci\'on, cu\'ales son sus funciones y sus responsabilidades, se obtiene la siguiente respuesta: \emph{Son las relacionadas con supervisi\'on de las registraciones contables de la empresa, conciliaciones de cuentas patrimoniales y de resultado� confecci\'on de balance mensual semestral y anual, pagos y presentaciones de impuestos y previsionales ante los organismos de control}. Adem\'as, muchas de las tareas actualmente asignadas a este departamento son delegadas hacia el departamento de Finanzas. 
	
Todos los departamentos est\'an constantemente pidiendo m\'as personal para poder llevar a cabo sus funciones aunque sean concientes de la situaci\'on actual de la empresa. A pesar de esto, el resto de los jefes de departamentos son consistentes en la opini\'on de que el Departamento de Administraci\'on  es uno de los que mas empleados tiene. La fusi\'on de \'este con el departamento de Finanzas podr\'ia ayudar a mejorar el funcionamiento del \'ultimo.

\subsection{Atraso en los sueldos}
	La gerenta de Marketing y Promoci�n describi\'o que la compa\~{n}\'ia se encuentra en un momento complicado ya que la empresa adeuda los sueldos de los \'ultimos dos meses.
	
	El atraso en los sueldos de los empleados provoca, como es de esperarse, un malestar palpable por parte de los empleados. Seg\'un el Gerente del Dpto. de Sistemas, \emph{Hay una disconformidad generalizada del recurso humano}. Esto genera que los trabajadores sientan que est\'an en una posici\'on inestable. Durante las encuestas se nos ha dicho que muchos de los empleados no trabajan durante el horario laboral y que existe una falta entusiasmo en el personal y tambi\'en falta de responsabilidad.
	
Todo esto deriva en la en un ambiente poco propicio para la evoluci\'on de la organizaci\'on.

\subsection{Recortes de personal}
	En los �ltimos dos a�os, la empresa se redujo aproximadamente en un 40\%. La causa del recorte fue la salida de Marsans del gerenciamiento de Aerol\'ineas Argentinas. Como consecuencia se produjo una reestructuraci\'on de las unidades de negocio dentro de la organizaci\'on que hizo necesaria la incorporaci\'on de personal con el objetivo de reactivar a la compa\~{n}\'ia en el mercado.

\subsection{Incorrecto manejo de pasantes}
	Las vacantes que se originaron como consecuencia del recorte en el personal de la empresa fueron llenadas con pasantes. Esto derivo en algunos problemas:
\begin{enumerate}
\item Necesidad de entrenamiento: Al contratar a gente sin experiencia fue necesario capacitarlos. Por lo tanto, en una etapa en la cual escaseaba el personal hubo que destinar recursos para el entrenamiento de los pasantes. En algunos casos este entrenamiento se vio imposibilitado porque no hab�a personal suficiente para llevarlo adelante.
\item Ubicaci\'on de pasantes en puestos que necesitaban de un profesional: Con el objetivo de reducir costos, se ubicaron a pasantes en puestos para los cuales no eran id\'oneos ya sea por falta de capacitaci\'on o por falta de experiencia. 
\end{enumerate}
	El uso de pasantes por parte de la organizaci\'on fue un parche frente a la dr\'astica reducci�n de personal de la empresa. 


\subsection{Efecto de la crisis econ�mica mundial}

	La crisis econ\'omica mundial fue un agravante a los conflictos que ya atravesaba Marsans. Esta crisis que comenz\'o en el 2007 y afect\'o las econom\'ias de todo mundo, afecto especialmente a Marsans ya que tuvo un fuerte impacto en la industria del turismo.

	Al ser entrevistada, la gerenta de Marketing y Promoci\'n afirm�: \emph{La crisis local y mundial, hizo que haya disminuido la cantidad de pasajeros. Viajar es un lujo y es uno de los aspectos que los argentinos recortan cuando hay crisis. La crisis nacional afecto directamente en el negocio de venta de paquetes tur\'isticos como de viajes dentro de la Argentina como hacia el exterior. La internacional golpeo nuestro principal mercado para el receptivo, el espa�ol, lo cual hizo que el negocio receptivo se vea seriamente afectado.}

\subsection{Falta de flexibilidad frente a las fluctuaciones del mercado}
	La organizaci\'on no demuestra ser flexible a los cambios en el contexto econ\'omico. Dentro del cuestionario formulado se preguntaba si se planeaba ofrecer nuevos productos o servicios en el corto plazo y si se hab\'ian producido cambios en los paquetes ofrecidos. Pese a que en el \'ultimo tiempo la empresa est\'a atravesado una etapa de crisis, no se han realizo modificaciones en los productos y servicios brindados y se continua con una misma pol�tica de producci\'on, ignorando la baja en la demanda.

\subsection{Desuso de los resultados del Departamento de Marketing}
	No se han utilizado las investigaciones de mercado para el dise\~{n}o y desarrollo de productos. La mayor parte de las tomas de decisiones se han hecho seg\'un intuici\'on de acuerdo a la mayor�a de los gerentes. 
Se puede rescatar lo siguiente del Gerente de Producto: \emph{El principal problema de la empresa es que nunca hubo una integraci�n vertical� es decir era por intuici�n de lo que se deb�a hacer}. 

	Este departamento no deber\'ia armar un producto en funci\'on de la intuici\'on de su gerente o en su defecto del Directo General. Para la venta de un producto, es necesario hacer una investigaci\'on de mercado. Esto demuestra la falta de comunicaci\'on que existe entre los departamentos y la falta de profesionalismo a la hora de la toma de decisiones. Tomando nuevamente la cita del Gerente de Administraci\'on: \emph{La persona a cargo de la empresa no cuenta con ninguna formaci�n profesional� tomando decisiones bajo caprichos}.

	Tambi\'en puede contribuir a esta situaci\'on el hecho de que el Departamento de Marketing no haga una correcto an\'alisis del mercado y se segmantaci\'on, tal como explica el Gerente de Sistemas.

\subsection{Desuso de los resultados del Departamento de Marketing}
	Se han tomado malas decisiones relacionadas con las inversiones en los paquetes armados para vender, tal como fu\'e el caso de la introducci\'on de los paquetes a China e India. \emph{En ambos destinos se invirti\'o dinero en marketing y publicidad no obteniendo casi ning�n retorno}, cita obtenida del Gerente de Sistemas que nuevamente da cuenta de la falta de profesionalismo en la toma de decisiones y los problemas de comunicaci\'on en los sectores. (Ver secci�n 1.7)

	Para que la empresa se recupere de su deteriorada situaci\'on econ\'omica, debe focalizarse en el desarrollo de pocos productos que generen flujo de dinero para que la empresa tenga ingresos. Con ese dinero, se deber\'ia pagar los sueldos adeudados a los empleados (Ver secci�n 1.2). Los productos elegidos deben ser pocos, de forma de poder focalizarse en la venta de los mismos. El departamento de producto cuenta con un gerente y 10 empleados a cargo que se encargan de Data Entry y operaciones de producto. Este n\'umero de empleados es muy grande dada la envergadura actual de la empresa.

\subsection{Imagen Comercial}

	El nombre de Marsans ha sido desprestigiado en los \'ultimos a\~{n}os debido a los problemas que tuvo durante el manejo de Aerol\'ineas Argentinas y su conflictivo desenlace. El pueblo Argentino tiene una muy mala imagen de la empresa y resulta casi imposible revertirla. Los clientes solamente hacen compras a corto plazo por las inseguridades que tienen sobre el futuro de la empresa. Es necesario un cambio de nombre para poder construir una imagen nueva para a la gente. De esa forma, no habr\'an prejuicios al momento de elegir una compa\~{n}\'ia para la adquisici\'on de un producto.
	
\end{document}