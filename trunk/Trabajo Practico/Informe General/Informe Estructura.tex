\documentclass[12pt,a4paper,spanish]{article} 
\usepackage{babel}
\usepackage [T1]{fontenc}
\usepackage [latin1]{inputenc}
\usepackage{graphicx}
\usepackage{array}
	  \oddsidemargin 0in
      \textwidth 6.75in
      \topmargin 0in
      \textheight 10.0in
      \parindent 0em
      \parskip 2ex
\usepackage{anysize}
\marginsize{3cm}{2cm}{1.0cm}{1.0cm}

\pagestyle{plain}

\begin{document} 
\title{
\begin{table}[!h]
	\begin{tabular}{m{2cm}m{15cm}}
		\multicolumn{1}{l}{}
		 \includegraphics[scale=0.25, bb=0 0 0 0]{Logo-fiuba.PNG} & 
		 \begin{center}
		 	\begin{LARGE}
				Universidad de Buenos Aires	\linebreak \linebreak		 							Facultad de Ingenier\'{i}a  \linebreak \linebreak
				7112 - Estrucutra de las Organizaciones \linebreak \linebreak
				2do. Cuatrimestre de 2009
			\end{LARGE}
		 \end{center}\\
\end{tabular}
\end{table}
\begin{Large}
 \begin{center}
		\underline{Trabajo Pr\'{a}tico} \linebreak \linebreak
        Grupo Nro:	R2
\end{center}
\end{Large}
}
\date{}
\maketitle

\thispagestyle{empty}
\author{
\begin{Large}
 \begin{center}
		\underline{Integrantes}  \linebreak 
\end{center}
\end{Large}
\begin{center}
	\begin{tabular}{|| c | c | c ||}
		\hline
		\begin{large}Apellido,Nombre\end{large} & 
		\begin{large}Padr\'{o}n Nro.\end{large} & 
		\begin{large}E-mail\end{large}\\
		\hline
		Fern\'{a}ndez Nicol\'{a}s  & 88.599 & nflabo@gmail.com\\
		\hline
		    &  & \\
		\hline
		   &  & \\
		\hline
		\end{tabular}
\end{center}
}
\newpage
\setcounter{page}{1} 
\tableofcontents
\newpage
\section{Alcances del Trabajo Pr\'{a}ctico}

\section{Tabla de Comparaci�n de Empresas}

\subsection{Marsans}
\subsubsection{Disponibilidad}
Nicol\'{a}s Fern\'andez, uno de los integrantes de nuestro grupo trabaj\'{o} en la empresa alrededor de dos a\~nos. Eso hizo que forme una buena relaci\'{o}n con algunos de los integrantes de la misma. Hablando con uno de los gerentes le coment\'{o} la idea de trabajar con Marsans Argentina S.A. a lo que el gerente le comunic\'{o} que no habr\'{i}a ning\'{u}n problema, que simplemente avisando con un peque\~no tiempo de anticipaci\'on, de no m\'{a}s de 2 o 3 d\'{i}as, podr\'{i}amos concretar las entrevistas o recavar la informaci\'on necesaria por medios electr\'onicos. Esto nos di\'{o} una pauta de que la disponibilidad es muy buena y que la empresa tiene la predisposici\'{o}n que necesitamos, aunque con una peque\~na demora, por lo cual el puntaje elegido fue un 8.

\subsubsection{Contacto}
Nicol\'{a}s Fern\'andez tiene contacto directo con: Sonia Kraimer (Gerencia de Recursos Humanos) y Alejandro Zubiria (Supervisor de Sistemas), con los cuales mantiene una muy buena relaci\'{o}n. A su vez conoce a distintas personas de otras \'{a}reas y tambi\'{e}n a trav\'{e}s de estas dos personas mencionadas puede obtener el contacto con cualquier otra persona que se necesite, en cualquier nivel jer\'arquico. Los contactos de la empresa nos parecieron muy predispuestos y creemos que vamos a poder acceder a cualquier contacto de otra \'{a}rea con lo cual el puntaje elegido fue un 10.

\subsubsection{Estructura}
La empresa es reconocida dentro del pa\'{i}s. La misma cuenta con un organigrama formal. Tiene diversos departamentos, sectores y puestos en los que se delegan responsabilididas y disminuye progresivamente la autoridad formal. Esto resulta una estructura ideal para el proyecto que realizaremos. El puntaje elegido fue un 10.

\subsubsection{Conocimiento de la Estructura}
Los contactos que tenemos en la empresa tienen amplio conococimiento acerca de la estructura, al igual que Nicol\'{a}s Fern\'andez, quien trabaj\'{o} all\'{i} en el \'area de Sistemas, que asiste al funcionamiento de todas las dem\'as, con lo cual se tiene un conocimiento bastante vasto sobre la misma y el hecho de que un integrante del grupo la conozca nos resulta muy \'{u}til. El puntaje elegido fue un 9.

\subsubsection{Ubicaci\'{o}n}
La empresa se encuentra en Suipacha y Avenida Santa F\'{e}, muy cerca del centro de la ciudad con lo cual se puede llegar cas\'{i} desde cualquier punto y queda relativamente cerca de ambas sedes de la facultad, tanto Las Heras como Paseo Col\'{o}n. A pesar de que no todos los integrantes vivimos en la Ciudad de Bs. As., estamos de acuerdo en que la ubicaci\'{o}n es excelente al estar en plena ciudad y tener muy f\'{a}cil acceso y abundantes medios de transporte desde y hacia cualquier ubicaci\'on. El puntaje elegido fue un 10.

\subsubsection{Tama\~{n}o}
La empresa cuenta con alrededor de 70 empleados, con lo cual el n\'umero se encuentra dentro del rango \'optimo sugerido por la c\'atedra y nos parece un buen n\'{u}mero al no ser muy grande ni muy chico. El puntaje elegido fue un 10.

\small
\begin{center}
\begin{tabular}{|| c | c | c | c | c | c | c ||}
\hline
\hline
T\'{e}rminos & Peso & Soft.Fact. & Marsans & Obras Ferr. & Sist.Ctrl.Adm. & Manuf. Avellaneda\\
\hline
Disponib.  & 10 & 7 & 8 & 6 & 8 & 8 \\
\hline
Contacto   & 10 & 8 & 10 & 6 & 10 & 10 \\
\hline
Estructura & 9 & 9 & 10 & 5 & 6 & 8 \\
\hline
Conoc. Estr. & 9 & 6 & 9 & 5 & 5 & 10 \\
\hline
Ubicaci\'{o}n & 8 & 10 & 10 & 8 & 7 & 5 \\
\hline
Tama\~{n}o   & 7 & 8 & 10 & 4 & 5 & 10 \\
\hline
\hline
Totales & - & 421 & 521 & 302 & 370 & 452 \\
\hline

\end{tabular}
\end{center}

\subsection{Justificaci\'{o}n de la selecci\'{o}n de la empresa}

Una vez hecha la tabla de comparaci\'{o}n con todos los items que cre\'imos importantes, la empresa que sum\'o el mayor puntaje fue Marsans Argentina S.A.
Al elegir los pesos de la tabla de comparaci\'{o}n, en lo que m\'as \'enfasis se puso fue tanto en la disponibilidad como el contacto, teniendo un segundo plano la estructura y el conocimiento de la estrucutra y siendo las caracter\'isticas con menor relevancia, la ubicaci\'on y el tama\~no.
Marsans obtuvo puntajes similares para casi todos los items a evaluar a diferencia de las dem\'{a}s empresas, en las cuales fueron muy variados los puntajes; algunas con puntajes muy altos para algunos items y muy bajos para otros y otras teniendo una regularidad de puntajes en todos los items, pero siendo estos bajos.
En s\'{i}ntesis Marsans Argentina S.A. nos pareci\'{o} la empresa que mejor se puede amoldar al trabajo que debemos realizar y creemos que vamos a poder tener toda la informaci\'{o}n que necesitemos de la misma cuando la requerramos.
A continuaci\'{o}n detallamos el porque de la elecci\'{o}n de los puntajes de cada item en particular, con una breve explicaci\'{o}n de cada uno.


En s\'{i}ntesis creemos que la empresa elegida es la mejor para lo que vamos a tener que trabajar. La misma cuenta con todo lo pedido y esperamos que la elecci\'{o}n de la misma confirme a lo largo del trabajo que es la opci\'on m\'as adecuada para el mismo, por todo lo planteado en este documento.

\section{Historia de la Empresa}
\section{Minutas de Reuni\'{o}n}
\section{Casos de An\'{a}lisis}
\subsection{Elavadores H\'{e}rcules}

\end{document}