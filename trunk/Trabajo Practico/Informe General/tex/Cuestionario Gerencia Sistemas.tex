\documentclass[12pt,a4paper,spanish]{article}
\usepackage[spanish]{babel}
\usepackage [T1]{fontenc}
\usepackage [latin1]{inputenc}
\usepackage{graphicx}
\usepackage{float}
\usepackage{array}
	  \oddsidemargin 0in
      \textwidth 6.75in
      \topmargin 0in
      \textheight 10.0in
      \parindent 0em
      \parskip 2ex
\usepackage{anysize}
\marginsize{3cm}{2cm}{1.0cm}{1.0cm}

\pagestyle{plain}

\begin{document}
\title{
\begin{table}[!h]
	\begin{tabular}{m{2cm}m{15cm}}
		\multicolumn{1}{l}{}
		 \includegraphics[scale=0.25, bb=0 0 0 0]{logo-fiuba.png} & 
		 \begin{center}
		 	\begin{LARGE}
				Universidad de Buenos Aires	\linebreak \linebreak		 							Facultad de Ingenier\'{i}a  \linebreak \linebreak
				7112 - Estrucutra de las Organizaciones \linebreak \linebreak
				2do. Cuatrimestre de 2009
			\end{LARGE}
		 \end{center}\\
\end{tabular}
\end{table}
\begin{Large}
 \begin{center}
		\underline{Entrevistas} \linebreak \linebreak
        Grupo Nro:	R2
\end{center}
\end{Large}
}
\date{}
\maketitle

\thispagestyle{empty}

\author{
\begin{Large}
\begin{center}
		\underline{Integrantes}  \linebreak 
\end{center}
\end{Large}
\begin{center}
	\begin{tabular}{|| c | c | c ||}
		\hline
		\begin{large}Apellido,Nombre\end{large} & 
		\begin{large}Padr\'{o}n Nro.\end{large} & 
		\begin{large}E-mail\end{large}\\
		\hline
		Bruno Tom\'as & 88.449 & tbruno88@gmail.com\\
		\hline
		Chiabrando Alejandra Cecilia & 86.863 & achiabrando@gmail.com\\
		\hline
		Fern\'{a}ndez Nicol\'{a}s  & 88.599 & nflabo@gmail.com\\
		\hline
		Invernizzi Esteban Ignacio & 88.817 & invernizzie@gmail.com\\
		\hline
		Medbo Vegard & \- & vegard.medbo@gmail.com\\
		\hline
		Meller Gustavo Ariel & 88.435 & gustavo\_meller@hotmail.com\\
		\hline
		Mouso Nicol\'as & 88.528 & nicolasgnr@gmail.com\\
		\hline
		Mu\~noz Facorro Juan Mart\'in & 84.672 & juan.facorro@gmail.com\\
		\hline
		Wolfsdorf Diego & 88.162 & diegow88@gmail.com\\
		\hline
	\end{tabular}
\end{center}
}

\newpage
\setcounter{page}{1}
\tableofcontents

\newpage
	\section{Gerencia de Sistemas: Zubiria Alejandro}

	\paragraph{Pregunta}
	 ?`Cu\'al es su formaci\'on?  ?`Corresponde a las competencias requeridas por el cargo?  ?`Mantiene actualizados sus conociemientos?
	\paragraph{Respuesta}
	Me recibi de Licenciado en Sistemas. Ademas realice la certificacion 	MCSE de Microsoft.
	Mi cargo excede las competencias requeridas actualmente por la empresa.
	\paragraph{Pregunta}
	 ?`Cu\'al es su experiencia laboral?
	\paragraph{Respuesta}
	Realice una consultoria en infraestructura de redes, soporte tecnico y tareas de administracion de servidores y redes en distintas empresas.
	Ademas realizo estas tareas terciarizado.
	Actualmente realizo estas tareas en Marsans.

	\paragraph{Pregunta}
	 ?`Cu\'al es su funci\'on en la empresa?
	\paragraph{Respuesta}
	Como mencione anteriormente soy responsable del area de sistemas. En si es el manejo de infraestructura tecnologica y el cumplimiento de los procesos del software de gestion.

	\paragraph{Pregunta}
	 ?`Cu\'al es su funci\'on?  ?`Cu\'ales son sus responsabilidades?
	\paragraph{Respuesta}
Responder desde la tecnologia a las necesidades de negocio de la empresa.
Mantener el correcto nivel de servicio de los recursos tecnologicos y comunicacionales.
Establecer los adecuados niveles de soporte a los usuarios, tanto de forma local como remota.
Trazar un plan de actualizacion tecnologica a mediano y largo plazo.
	
	\paragraph{Pregunta}
	 ?`Qui\'enes son sus superiores? 
	\paragraph{Respuesta}
A el director general.
	\paragraph{Pregunta}
	 ?`A qui\'en le reporta?
	\paragraph{Respuesta}
Al director General y a el sector tecnologico de Marsans Espana.
	\paragraph{Pregunta}
	 ?`Qui\'enes son sus subordinados?  ?`Qu\'e tareas delega sobre ellos?
	\paragraph{Respuesta}
Actualmente ninguno.

	\paragraph{Pregunta}
	 ?`Delega tareas sobre personas de otras \'areas?
	\paragraph{Respuesta}
Si, se delegan. Por ejemplo sobre los data entry del area de producto, sobre el chequeo de la administracion de la base de datos.
Ademas se cuenta con la delegacion del soporte de software de gestion a la empresa que lo programo.
Se delega la parte de comunicacion a una persona terciarizada.

	\paragraph{Pregunta}
	 ?`Se le presentan los mismos problemas a diario o con cierta periodicidad?  ?`Cu\'ales son y c\'omo los resuelve?
	\paragraph{Respuesta}
Si, suelen ocurrir requerimientos particulares de forma repetitiva para los cuales existen procedimientos de respuesta, tales como bases de conocimientos, manuales de procedimientos o tambien acuerdo de niveles de servicio.

	\paragraph{Pregunta}
	 ?`En base a tipo de informaci\'on realiza sus decisiones?  ?`De d\'onde la obtiene?
	\paragraph{Respuesta}
	Depende de que decisiones se estan analizando, pero en particular desde la tecnologia el analisis de los procesos de la empresa, la estimacion de crecimiento, la estrategia comercial definida por la empresa.

	\paragraph{Pregunta}
	 ?`Qu\'e informaci\'on debe reportar?
	\paragraph{Respuesta}
Al no ser una empresa netamente tecnologica el reporte esta orientado a inversion vs resultado, entonces debe justificar y reportar de alguna forma la influencia que tuvo el presupuesto establecido para el area sobre los procesos de negocio de la empresa.

	\paragraph{Pregunta}
	 ?`Cu\'al es su relaci\'on y c\'omo se comunica con los dem\'as gerentes?
	\paragraph{Respuesta}
Se realiza mediante los lineamientos prefijados, es decir, en el caso de esta area se respetan los procedimientos.
	\paragraph{Pregunta}
	 ?`De la producci\'on de qu\'e \'areas depende la concreci\'on de la propia? (Interdependencia operativa)
	\paragraph{Respuesta}
Se depende indirectamente del area comercial para poder justificar la inversion realizada.

	\paragraph{Pregunta}
	 ?`Considera que existe alg\'un problema con la estructura de la empresa o la organizaci\'on actual de su gerencia?  ?`Tiene la posibilidad de cambiarla?  ?`Planea hacerlo?
	\paragraph{Respuesta}
Si, considero que existen varios problemas, entre los cuales se pueden mencionar los siguientes:
\begin{itemize}
\item Por escasez de recursos el nivel de respuesta no es el adecuado.
\item Los proyectos de inversion tecnologica estan frenados.
\item No se esta cumpliendo el presupuesto estimado para el area.
\end{itemize}
En este momento de la empresa no esta en mis manos poder cambiarlo.
El nivel de recursos es muy bajo y realmente el presupuesto esta totalmente vedado.

	\paragraph{Pregunta}
	 ?`Existe alg\'un tipo de evaluaci\'on de desempe\~{n}o?  ?`Cada cu\'anto tiempo se hace?  ?`Qu\'e aspectos se eval\'uan?  ?`Se encuentra conforme con su \'ultima evaluaci\'on recibida?
	\paragraph{Respuesta}
Por lo mencionado anteriormente no exite evualuacion de desempeno.
	\paragraph{Pregunta}
	 ?`Existe alg\'un tipo sistema de bonos?  ?`C\'omo funciona?  ?`C\'omo se determina lo que le corresponde a cada empleado?
	\paragraph{Respuesta}
En la parte de sistemas no se gestionan bonos.

	\paragraph{Pregunta}
	 ?`Considera adecuada la estructura de las dem\'as gerencias?  ?`Observa disfunciones?
	\paragraph{Respuesta}
No, no lo considero.
Los detalles del porque son los siguientes:
\begin{itemize}
\item La jefatura del area administrativa financiera esta sobredimensionada, es decir tienen 3 gerentes los cuales no son necesarios para la actualidad de la empresa.
\item No se cumple correctamente el proceso de facturacion y cobranza.
\item Es ineficiente el prodecimiento de conciliacion de cuentas.
\item No se hace el correcto analisis del mercado y su segmentacion.
\item La carga de datos es ineficiente.
\item No es correcto el enfoque de marketing de la empresa, es decir a donde se dirijen las publicidades.
\item Se recarga trabajo operativo y administrativo a los vendedores.
\item El area operativa funciona de forma desornada.
\item No estan bien definidas las responsabilidades y funciones de cada puesto de trabajo.
\item No existe un producto donde la empresa sea lider.
\item No cuenta la empresa con alguna certificacion, como por ejemplo podria ser la Q de calidad.
\item Procedimientos bastante informales.
\item Poca documentacion de los procedimientos de cada area que se podria estandarizar.
\item Disconformidad generalizada del recurso humano.
\end{itemize}

	\paragraph{Pregunta}
	 ?`Considera que a su sector le sobra o le falta personal para realizar sus tareas?
	\paragraph{Respuesta}
	Por lo menos faltaria una persona para realizar de forma correcta y operativa las tareas del area, pero por reduccion de personal no se puede contar con la misma.

\subsection{Comentarios adicionales: Evolucion de la empresa en los ultimos tiempos}
Hasta el 2001 durante la convertibilidad la empresa manejaba un monoproducto que era la venta de vuelos charter con hoteleria al caribe.
Ante la salida de la convertibilidad se tuvo que cambiar el negocio dado que dejo de ser rentable, mas que nada ejecutable.
Al pasar esto la empresa se puso como objetivo ampliar sus operaciones a una empresa mayorista de tipo emisivo y receptivo.
Se abren los departamentos de ventas internacionales,nacionales y receptivo.
Evolucionaron positivamente hasta el principios del 2008 donde comenzo la crisis intitucional dentro del pais y comenzaron a sentirse los efectos de la crisis financiera mundial.
La crisis nacional afecto directamente en el negocio de venta de paquetes turisticos como de viajes dentro de la Argentina como hacia el exterior.
La internacional golpeo nuestro principal mercado para el receptivo, el espanol, lo cual hizo que el negocio receptivo se vea seriamente afectado.
La empresa entro en un proceso de reestructuracion reduciendo personal y recursos, pasando de 120 personas a alredor de 60.
La caida abrupta en las ventas genero un problema financiero llevando a la empresa a la necesidad de vender en algunos casos por debajo de los costos con la intencion de obtener liquidez. Para ello se potencio el negocio de venta de pasajes areo.
Algunos de los negocios que se trataron de incursionar pero no tuvieron exito fueron:
Desarrollar el mercado hacia China y la India.
En ambos destinos se invirtio dinero en marketing y publicidad no obteniendo casi ningun retorno.
En el proceso de reduccion ademas se pasaron de 2 plantas a 1 sola, reduciendo considerablemente el area de trabajo.

\end{document}