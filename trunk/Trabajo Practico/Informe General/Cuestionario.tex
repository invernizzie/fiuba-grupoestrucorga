\documentclass[12pt,a4paper,spanish]{article}
\usepackage[spanish]{babel}
\usepackage [T1]{fontenc}
\usepackage [latin1]{inputenc}
\usepackage{graphicx}
\usepackage{float}
\usepackage{array}
	  \oddsidemargin 0in
      \textwidth 6.75in
      \topmargin 0in
      \textheight 10.0in
      \parindent 0em
      \parskip 2ex
\usepackage{anysize}
\marginsize{3cm}{2cm}{1.0cm}{1.0cm}

\pagestyle{plain}

\begin{document}
\title{
\begin{table}[!h]
	\begin{tabular}{m{2cm}m{15cm}}
		\multicolumn{1}{l}{}
		 \includegraphics[scale=0.25, bb=0 0 0 0]{logo-fiuba.png} & 
		 \begin{center}
		 	\begin{LARGE}
				Universidad de Buenos Aires	\linebreak \linebreak		 							Facultad de Ingenier\'{i}a  \linebreak \linebreak
				7112 - Estrucutra de las Organizaciones \linebreak \linebreak
				2do. Cuatrimestre de 2009
			\end{LARGE}
		 \end{center}\\
\end{tabular}
\end{table}
\begin{Large}
 \begin{center}
		\underline{Entrevistas} \linebreak \linebreak
        Grupo Nro:	R2
\end{center}
\end{Large}
}
\date{}
\maketitle

\thispagestyle{empty}

\author{
\begin{Large}
\begin{center}
		\underline{Integrantes}  \linebreak 
\end{center}
\end{Large}
\begin{center}
	\begin{tabular}{|| c | c | c ||}
		\hline
		\begin{large}Apellido,Nombre\end{large} & 
		\begin{large}Padr\'{o}n Nro.\end{large} & 
		\begin{large}E-mail\end{large}\\
		\hline
		Bruno Tom\'as & 88.449 & tbruno88@gmail.com\\
		\hline
		Chiabrando Alejandra Cecilia & 86.863 & achiabrando@gmail.com\\
		\hline
		Fern\'{a}ndez Nicol\'{a}s  & 88.599 & nflabo@gmail.com\\
		\hline
		Invernizzi Esteban Ignacio & 88.817 & invernizzie@gmail.com\\
		\hline
		Medbo Vegard & \- & vegard.medbo@gmail.com\\
		\hline
		Meller Gustavo Ariel & 88.435 & gustavo\_meller@hotmail.com\\
		\hline
		Mouso Nicol\'as & 88.528 & nicolasgnr@gmail.com\\
		\hline
		Mu\~noz Facorro Juan Mart\'in & 84.672 & juan.facorro@gmail.com\\
		\hline
		Wolfsdorf Diego & 88.162 & diegow88@gmail.com\\
		\hline
	\end{tabular}
\end{center}
}

\newpage
\setcounter{page}{1}
\tableofcontents

\newpage
\section{Cuestionario General}
En esta primera secci\'on se presentan las preguntas relacionadas con las generalidades de la empresa, que permiten obtener de primera mano informaci\'on sobre la direcci\'on, administraci\'on, productos y estructura de la empresa. M\'as adelante se indagar\'a sobre cada \'area en particular.

	\subsection{Generalidades}

	\paragraph{Pregunta}
	 ?`Cu\'al es la antig\"{u}edad de la empresa en el pa\'is?  ?`Y la del grupo Marsans?
	\paragraph{Respuesta}
	En el pa\'is Marsans tiene 40 a\~{n}os. En el mundo 101 a\~{n}os.

	\paragraph{Pregunta}
	 ?`De qu\'e forma se despliega geogr\'aficamente?  ?`Cu\'ales son sus sedes, filiales, puntos de venta?  ?`Est\'a presente en todo el pa\'is?
	\paragraph{Respuesta}
	Tiene su casa central en Capital Federal y sucursales en Rosario y C\'ordoba. Tambi\'en representaciones en Mar del Plata, Santa Fe y Neuquen.

	\paragraph{Pregunta}
	 ?`En qu\'e rama de la industria se ubica?  ?`Cu\'al es su participaci\'on en el mercado?  ?`Qu\'e empresas representan la competencia?
	\paragraph{Respuesta}
	Marsans es un operador mayorista de turismo. Es uno de los l\'ideres a nivel nacional e internacional en operaci\'on mayorista. La competencia est\'a conformada por Julia, Eurovips, Eurotour, Piamonte, entre otros.

	\subsection{Organigrama}
	A continuaci\'on se presenta el organigrama completo de la empresa, y se lo clasifica seg\'un los conceptos vistos en clase.

	\begin{figure}[H]
	\centering
	\includegraphics[scale=0.45]{organigrama.png}
	\end{figure}

	\paragraph{Pregunta}
	 ?`Qu\'e cantidad de personal hay en el nivel dirigencial?  ?`En el operativo?  ?`En el de supervisi\'on? PRESENTAR UNA DISTRIBUCION PORCENTUAL
	\paragraph{Respuesta}
	A nivel Dirigencial, hay 3 personas. En la parte operativa 46. En el de supervision 8.

	\paragraph{Pregunta}
	 ?`Qu\'e cantidad de personal es efectivo, y cu\'al contratado? CALCULAR RELACION
	\paragraph{Respuesta}
	Todo el personal de la compaa es efectivo.

	\paragraph{Pregunta}
	 ?`C\'omo afectan las fluctuaciones de la actividad a la incorporaci\'on o desvinculaci\'on de personal?
	\paragraph{Respuesta}
	El personal suele mantenerse estable para las temporadas altas y bajas, s\'olo que en las altas, suele contratarse personal de back up,  debido a la mayor demanda de productos.

	\subsection{Direcci\'on}

	\paragraph{Pregunta}
	 ?`Cu\'al es el objetivo general de la empresa?
	\paragraph{Respuesta}
	El objetivo de la empresa es continuar siendo una multidestino (se comercializan todo tipo de destinos) y rentabilizar cada vez mas cada venta.

	\paragraph{Pregunta}
	 ?`Cu\'ales son los pilares de la estrategia corporativa?
	\paragraph{Respuesta}
	Brindar un excelente servicio y calidad en la atenci\'on, para diferenciarnos de la competencia.

	\paragraph{Pregunta}
	 ?`Ha atravesado la empresa un proceso de dise\~{n}o organizacional formal?  ?`Es actualizado mediante cambios organizacionales? Caso positivo,  ?`qu\'e tan frecuentemente?
	\paragraph{Respuesta}
	Se ha realizado una restructuraci\'on del sector nacional, luego de la salida de Marsans del gerenciamiento de Aerolineas Argentinas. Esto ha hecho que algunas unidades de negocio se fusionen con otras.

	\paragraph{Pregunta}
	 ?`Cu\'ales son los mecanismos de toma de decisi\'on y determinaci\'on de estrategias utilizados?
	\paragraph{Respuesta}
	Las tomas de decisi\'on a nivel institucional, pasan por presidencia, con el consenso previo de los accionistas en Madrid. Las operativas son realizadas por cada gerente o supervisor de departamento.

	\paragraph{Pregunta}
	 ?`Cu\'al es el grado de participaci\'on en las decisiones de los distintos niveles jer\'arquicos? DETALLAR DECISIONES TOMADAS EN CADA NIVEL
	\paragraph{Respuesta}
	Todo el nivel de supervisi\'on y general tiene la misma injerencia en la toma de decisiones.

	\subsection{Producci\'on}

	\paragraph{Pregunta}
	 ?`Qu\'e productos o servicios ofrece la empresa al mercado?
	\paragraph{Respuesta}
	Marsans es un multiproducto, ofrece paquetes tur\'isticos nacionales e internacionales a agencias minoristas de turismo, as\'i como tambien ofrece servicios receptivos para pasajeros extranjeros.

	\paragraph{Pregunta}
	 ?`Alcanzan estos productos el \'exito perseguido?
	\paragraph{Respuesta}
	Si, aunque la crisis local y mundial, hizo que haya disminuido la cantidad de pasajeros. Viajar es un lujo y es uno de los aspectos que los argentinos recortan cuando hay crisis. No han tenido el mismo resultado  los paquetes corporativos, que siguen manteniendo su nivel.

	\paragraph{Pregunta}
	 ?`Planea ofrecer en el corto plazo nuevos productos o servicios?
	\paragraph{Respuesta}
	No, por ahora los que manejamos cada temporada.

	\paragraph{Pregunta}
	 ?`Se discontinuaron en el \'ultimo tiempo productos o servicios?  ?`Por qu\'e?
	\paragraph{Respuesta}
	No, por ahora seguimos ofreciendo los mismos, aunque con menos pasajeros.

	\paragraph{Pregunta}
	 ?`Se terceriza alguna parte del proceso de producci\'on?  ?`Por qu\'e? (disminuir costos, abastecer la demanda sin aumentar el tama\~{n}o de la empresa, etc.)
	\paragraph{Respuesta}
	Se terceriza la limpieza, seguridad  y la mensajeria. 

	\subsection{Gerencia de Marketing y Promoci\'on}

	\paragraph{Pregunta}
	 ?`Cu\'al es su formaci\'on?  ?`Corresponde a las competencias requeridas por el cargo?  ?`Mantiene actualizados sus conociemientos?
	\paragraph{Respuesta}
	Dise\~{n}ador gr\'afico, se corresponde con las competencias requeridas y mantengo mis conocimientos actualizados.

	\paragraph{Pregunta}
	 ?`Cu\'al es su experiencia laboral?
	\paragraph{Respuesta}
	Director de Cuentas MCW (Publicidad)

	\paragraph{Pregunta}
	 ?`Cu\'al es su funci\'on en la empresa?
	\paragraph{Respuesta}
	Estoy a cargo de la gerencia de Marketing y Promoci\'on de Marsans en Argentina, Brasil (hasta 2007) y Chile.
	
	\paragraph{Pregunta}
	 ?`Cu\'al es su funci\'on?  ?`Cu\'ales son sus responsabilidades?
	\paragraph{Respuesta}
	Me encargo de comunicar y dise\~{n} la imagen de la empresa frente a sus clientes directos (agencias).
	
	\paragraph{Pregunta}
	 ?`Qui\'enes son sus superiores? 
	\paragraph{Respuesta}
	La gerencia general.

	\paragraph{Pregunta}
	 ?`A qui\'en le reporta?
	\paragraph{Respuesta}

	\paragraph{Pregunta}
	 ?`Qui\'enes son sus subordinados?  ?`Qu\'e tareas delega sobre ellos?
	\paragraph{Respuesta}
	Dos equipos de trabajo: MKT, encargado de im\'agen y comunicaci\'on, y Promoci\'on, a cargo del contacto con el trade.

	\paragraph{Pregunta}
	 ?`Delega tareas sobre personas de otras \'areas?
	\paragraph{Respuesta}

	\paragraph{Pregunta}
	 ?`Se le presentan los mismos problemas a diario o con cierta periodicidad?  ?`Cu\'ales son y c\'omo los resuelve?
	\paragraph{Respuesta}

	\paragraph{Pregunta}
	 ?`En base a tipo de informaci\'on realiza sus decisiones?  ?`De d\'onde la obtiene?
	\paragraph{Respuesta}
	Las decisiones se toman en base a las necesidades de comunicaci\'on que tenga la empresa. La informaci\'on se obtiene a traves del estudio de mercado, y de otras \'areas de la empresa.
	
	\paragraph{Pregunta}
	 ?`Qu\'e informaci\'on debe reportar?
	\paragraph{Respuesta}
	Informaci\'on comercial, interna y estudio de la competencia


	\paragraph{Pregunta}
	 ?`Cu\'al es su relaci\'on y c\'omo se comunica con los dem\'as gerentes?
	\paragraph{Respuesta}
	La comunicaci\'on con otros gerentes es a trav\'es de reuniones peri\'odicas dentro de la empresa.

	\paragraph{Pregunta}
	 ?`De la producci\'on de qu\'e \'areas depende la concreci\'on de la propia? (Interdependencia operativa)
	\paragraph{Respuesta}
	Interactuamos bastante con el Departamento de Producto y de ventas para la realizaci\'on de la comunicaci\'on.

	\paragraph{Pregunta}
	 ?`Considera que existe alg\'un problema con la estructura de la empresa o la organizaci\'on actual de su gerencia?  ?`Tiene la posibilidad de cambiarla?  ?`Planea hacerlo?
	\paragraph{Respuesta}

	\paragraph{Pregunta}
	 ?`Existe alg\'un tipo de evaluaci\'on de desempe\~{n}o?  ?`Cada cu\'anto tiempo se hace?  ?`Qu\'e aspectos se eval\'uan?  ?`Se encuentra conforme con su \'ultima evaluaci\'on recibida?
	\paragraph{Respuesta}

	\paragraph{Pregunta}
	 ?`Existe alg\'un tipo sistema de bonos?  ?`C\'omo funciona?  ?`C\'omo se determina lo que le corresponde a cada empleado?
	\paragraph{Respuesta}

	\paragraph{Pregunta}
	 ?`Considera adecuada la estructura de las dem\'as gerencias?  ?`Observa disfunciones?
	\paragraph{Respuesta}

	\paragraph{Pregunta}
	 ?`Considera que a su sector le sobra o le falta personal para realizar sus tareas?
	\paragraph{Respuesta}

\subsection{Recursos Humanos}
	\paragraph{Pregunta}
	 ?`Cu\'antas personas est\'an destinadas a este \'area?  ?`Qu\'e proporci\'on representa del total de la empresa?
	\paragraph{Respuesta}
	Hay una sola persona en el departamento de RRHH que es la responsable del mismo.

	\paragraph{Pregunta}
	 ?`Est\'an especificamente capacitados en el \'area de RR.HH.?
	\paragraph{Respuesta}
	Si, la responsable es Psic\'olaga y cuenta con un posgrado en Organizaci\'on y Conducci\'on de RRHH, dictado por la facultad de Psicologia de la UBA.

	\paragraph{Pregunta}
	 ?`Qui\'en pide (o est\'a autorizado a pedir) la incorporaci\'on de personal?  ?`Qui\'en decide cu\'ando reclutar nuevo personal?
	\paragraph{Respuesta}
	La incorporacin de personal debe ser autorizada por el Director, el define cu\'ando y c\'omo.
	
	\paragraph{Pregunta}
	 ?`Cu\'ales son los m\'etodos usuales de reclutamiento de personal?  ?`Involucran personal de otras \'areas?
	\paragraph{Respuesta}
	Se recluta personal a trav\'es de avisos en los medios del Trade, a trav\'es de paginas web especializadas (Bumeran, Computrabajo) y en algunos casos a traves de consultoras.
	
	\paragraph{Pregunta}
	 ?`Hay pol\'iticas de bono?  ?`Cu\'ales?
	\paragraph{Respuesta}
	Existe un plus que se abona a las unidades de negocio que alcazaron mensualmente los objetivos del \'area. Se llama PRV y consiste en el 8,33\% del salario b\'asico.
	
	\paragraph{Pregunta}
	 ?`Hay plan de carrera?  ?`Incluye a todo el personal?
	\paragraph{Respuesta}
	En este momento se esta desarrollando.
	
	\paragraph{Pregunta}
	 ?`Se capacita al personal?  ?`Dentro o fuera de la empresa?  ?`Qui\'en se encarga de hacerlo?
	\paragraph{Respuesta}
	Las capacitaciones se realizan in company y fuera de la empresa. Se contratan consultoras para realizar la misma, los proveedores suelen tambien capacitar al personal. 


\end{document}