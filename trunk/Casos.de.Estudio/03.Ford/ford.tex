\section{Introducci\'on}
	Para el \'exito de una empresa, es indispensable que \'esta conozca su ambiente y se sepa adaptar al mismo.
	No basta \'unicamente con cerrar tratos convenientes con los clientes, sino que tambi\'en hace falta considerar otras variables, como por ejemplo los proveedores y la competencia.
	El conjunto de todas estas variables forma lo que se conoce como el ambiente.

	Formalmente, el ambiente se puede definir como el sistema de nivel superior en el que se inserta la empresa y est\'a formado por aquellos factores, fuerzas o variables que influyen directa, o indirectamente, en los procesos de gesti\'on\cite{ITE}.
	El conocimiento de una empresa de su ambiente le permite explotar al m\'aximo su potencial e incluso poder anticiparse a situaciones riesgosas.

	La empresa Ford Motor Company  se fund\'o a principios del siglo XX.
	Ya desde sus comienzos se dedic\'o a la producci\'on de autom\'oviles.
	Mediante la instalaci\'on de cintas de ensamblaje r\'apidamente consigui\'o volverse altamente competitiva y de esta forma se hizo un lugar en el mercado.

\section{Cambios en la administraci\'on del ambiente}
	\begin{enumerate}
		\item{\emph{Suministro de partes por parte de proveedores:}} se agregaba valor mediante el ensamblaje de las piezas.
		\item{\emph{Integraci\'on vertical:}} comienzo de fabricaci\'on de las piezas necesarias previas al ensamblaje.
		\item{\emph{Potenciales proveedores m\'as eficientes:}} fin de modelo verticalmente integrado. Vuelta a comprar las piezas a proveedores eficientes.
		\item{\emph{Irrupci\'on en el mercado por parte de competencia externa:}} adquisici\'on de porcentajes en las empresas de proveedoras.
	\end{enumerate}

\section{An\'alisis}
	\subsection{Suministro de partes por parte de proveedores}
		La innovaci\'on de la Ford Motor Company en sus inicios, fue la aplicaci\'on de una cinta de ensamblaje m\'ovil.
		La utilizaci\'on de esta cinta le permitía tener una capacidad de producci\'on muy grande, que no necesariamente podía cubrir con la propia producci\'on de los insumos que necesitaba.
		No depender de los proveedores, por los problemas de calidad e incompatibilidad entre proveedores que ellos conllevan, no era tan importante como captar el mayor porcentaje del mercado automotriz incipiente.
		Por esta raz\'on, se acord\'o con determinados proveedores para poder obtener los insumos necesarios para satisfacer la demanda en el mercado.

		El problema que surgía de este modelo era que el proceso de producci\'on se hacía muy largo.
		Cada vez que las piezas de un proveedor eran incompatibles con las de otro proveedor, se demoraba la producci\'on hasta hacer los ajustes necesarios que permitieran la compatibilidad entre las piezas.
		Era necesario tener la producci\'on fuertemente estandarizada para evitar estas situaciones y tener un documento en el cual ampararse al momento de hacer los reclamos a los proveedores: los contratos de suministro.

		No obstante, para explotar la ventaja comparativa que tenía por sobre otras empresas automotrices, la tecnología de la cinta de ensamblaje m\'ovil, era necesario un manejo del ambiente con características similares a estas.

	\subsection{Integraci\'on vertical}
		Una vez ya consolidada en el mercado, la Ford Motor Company crey\'o prioritario no descuidar la calidad de sus productos.
		Al no poder cambiar directamente la calidad de sus insumos ya que estos eran producidos por sus proveedores, decidi\'o fabricarlos por sí misma.
		Esta integraci\'on vertical se realiz\'o de dos formas: por un lado, se tom\'o el control de las empresas de algunos proveedores y por otro, se comenzaron operaciones propias para la producci\'on de los insumos necesarios.
		Así fue que, por ejemplo, compr\'o minas para la extracci\'on de hierro, lo transport\'o a sus plantas donde fabric\'o acero y luego lo us\'o para la manufactura de la carrocería y otras piezas.

		La integraci\'on vertical le permitía no depender de otras organizaciones para satisfacer la demanda del mercado.
		La posici\'on de Ford ahora le permitía poder satisfacer por sí misma un aumento en la demanda.
		Adem\'as, había alcanzado un poder adquisitivo que le permitía tomar el control de otras empresas.

		No es un detalle menor la depresi\'on econ\'omica que atraves\'o Estados Unidos en la d\'ecada del '30, la cual probablemente provoc\'o la quiebra de algunos de los proveedores, haciendo que estos no puedan cumplir los compromisos que tenían hacia Ford. 
		Para esta empresa, la integraci\'on vertical era una salida para no verse arrastrada junto a algunos de sus proveedores.

	\subsection{Potenciales proveedores m\'as eficientes}
		Sin embargo, la producci\'on de sus insumos result\'o ser demasiado costosa.
		El excesivo tama\~no que había adquirido Ford hacía que aparezcan problemas de comunicaci\'on.
		Mientras tanto, aparecían potenciales proveedores que demostraban ser m\'as eficientes en la producci\'on de los insumos que Ford necesitaba, dando lugar a una oportunidad para reducir los costos de los insumos.
		Adem\'as, la situaci\'on econ\'omica mundial había cambiado nuevamente y en la d\'ecada del '50 se esperaba nuevamente una \'epoca de estabilidad.

		Por estas razones, Ford decidi\'o volver a acordar con proveedores para el suministro de materia prima.
		La alta participaci\'on que tenia Ford en el mercado le permiti\'o poder negociar precios favorables al momento de establecer las condiciones del acuerdo.
		De esta forma logr\'o obtener ventajas sobre General Motors que seguía con un modelo basado en la integraci\'on vertical y sufría problemas similares a los de Ford.

	\subsection{Irrupci\'on en el mercado por parte de competencia externa}
		Durante varias d\'ecadas, la competencia que sufría Ford era solo competencia interna por parte de Chrysler y General Motors.
		Esta competencia no era destructiva ya que las empresas acordaban informalmente políticas de precios, estableciendo de esta manera un oligopolio en la industria automotriz.
		Cada una de las empresas tenía un sector del mercado y su porcentaje dentro de \'este no sufría grandes variaciones.
		Sin embargo, en la d\'ecada de 1980, la industria japonesa logr\'o ser suficientemente competitiva para ingresar en el mercado estadounidense.

		Las empresas japonesas contaban con determinadas t\'ecnicas que les permitían obtener ventajas comparativas sobre las estadounidenses:
		\begin{itemize}
			\item{\emph{Participaci\'on en los proveedores:}}
				Mediante la tenencia de acciones de las empresas proveedoras, las empresas japonesas lograban incidir en el control de estas sin llegar a tener los problemas de comunicaci\'on y escala que Ford había sufrido cuando se había integrado verticalmente.

			\item{\emph{Sistema de Inventario Justo a Tiempo (JIT):}}
				Es una t\'ecnica implementada inicialmente por Toyota que consiste en reducir el costo de la gesti\'on y reducir las p\'erdidas que ocurren debido a contar con stocks innecesarios.
				No se produce bajo suposiciones, sino bajo pedidos concretos.
				Por lo tanto, se reducen los costos de almacenamiento.
				La aplicaci\'on de esta t\'ecnica ha generado un aumento en la productividad de las empresas que la utilizaron.

			\item{\emph{Keiretsu:}}
				El keiretsu es un tipo de modelo empresarial donde existe una coalici\'on de empresas unidas bajo determinados intereses.
				La ventaja de este tipo de modelo es que en estas coaliciones suele haber organizaciones que pueden ayudar en la parte financiera de la producci\'on, como por ejemplo bancos.
				Un ejemplo de Keiretsu es Mitsubishi, que actualmente maneja el banco Mitsubishi UFJ.

			\item{\emph{Investigaci\'on:}}
				La investigaci\'on se realizaba entre varias empresas y con la colaboraci\'on del gobierno.
				De esta forma, la investigaci\'on producía, con menores costos individuales, mayores resultados que, a pesar de tener que ser compartidos entre algunas empresas, permitían obtener una ventaja comparativa sobre las empresas que no formaban parte del grupo de colaboraci\'on.

		\end{itemize}

		Estas t\'ecnicas fueron, algunas en mayor medida que otras, copiadas por Ford.
		Una de las medidas al adoptar estas t\'ecnicas fue comprar acciones de algunos de sus proveedores (Cummings, Excel Industries y Decona).
		Como ya se mencion\'o anteriormente, esto le permiti\'o tener una cuota de control sobre el accionar de sus proveedores.
		Otra medida fue diversificarse en el mercado, logrando comprar el 49$\%$ de la compa\~nia de alquiler de autos Hertz, donde adem\'as de obtener ganancias por medio del negocio de la empresa en sí, consigui\'o publicidad y colocaci\'on de sus autos ya que gran parte de los que se alquilan, son producidos por la Ford Motor Company.
		En lo que respecta a la investigaci\'on, Ford se asoci\'o con otros competidores para financiar un programa de investigaci\'on conjunta y desarrollar nuevos productos.

\section{Paralelismo entre Ford y las empresas japonesas}
Como mencionamos anteriormente uno de los problemas que le aparecen a Ford es la irrupci\'on en el mercado por parte de competencia externa, en este caso empresas japonesas. A partir de ahora detallamos el paralelismo entre lo hecho por Ford y lo hecho por las empresas japonesas, para intentar ver las diferencias entre ambos.
	\subsection{Visi\'on adelantada}
	Las empresas japonesas fueron precursoras y mediante un estudio de mercado implementaron diversas ideas las cuales le resultaron muy favorables. La gran ventaja es que lo hicieron antes que todo el resto de las empresas mundiales con lo cual se les anticipar\'on y les ganaron terreno. En esto tambi\'en le ganan a Ford, ya que es una empresa de Estados Unidos. 
	\subsection{Diferencias de pensamiento}
	Las f\'abricas japonesas se aliaban con los proveedores, ya que tenían el pensamiento de que si a una le iba bien a la otra tambi\'en le iba a ir bien. Con una buena cooperaci\'on entre ambas partes, ambas crecerían a la par. En Norteam\'erica la relaci\'on entre proveedores y f\'abricas no era así. Se tenía un pensamiento de que había que tratar de obtener la mayor ventaja posible sobre el otro, es decir, cuanto m\'as se le podía sacar al otro mejor. Este es un pensamiento totalmente erroneo y es uno de los puntos m\'as importantes en los cuales los japoneses ganaron terreno. Supieron dejar de lado diferencias y rencores entre empresas dandose cuenta de los beneficios que les podía traer y en eso fueron muy inteligentes.
	\subsection{Implementaci\'on err\'onea de los m\'etodos japoneses}
	Al ver como en Jap\'on las cosas funcionaban bien, en Norteam\'erica se los quizo imitar. Las f\'abricas Norteam\'ericanas intentaron aliarse con los proveedores, pero no como debían. En vez de producir una alianza terminaban comprando a los proveedores con lo cual se produj\'o una integraci\'on vertical, lo cual no les servía absolutamente para nada, no mejoraban ya que tenían m\'as cosas para producir y no se dedicaban precisamente a lo que ellos producían. Ahí se ve el error grave cometido. Si se quiere imitar algo, se debe hacer en forma correcta, no a las apuradas y mal.
	\subsection{Diferencias culturales}
	Para la gente de Jap\'on es un orgullo ver que la empresa en la que uno trabaja le va bien. Los japoneses tienen una forma distinta de ver las cosas. Se pueden quedar horas extra trabajando en conjunto para mejorar sus empresas, sin importarles que les den o no dinero extra. El orgullo que ellos tienen por ver que lo que hacen sale bien, es una gran satisfacci\'on. En Norteam\'erica (y en gran parte del mundo) no sucede así. Los empleados no ven la hora de que termine su día de trabajo, no les interesa que la empresa crezca (hasta algunos quieren que a su empresa le vaya mal, sin darse cuenta de que su empleo depende de ello) y por cada segundo extra que trabajen van a exigir que se les pague. Hay como una ley de que si no se les paga extra no se trabaja extra. Esto es una de las cosas m\'as dificiles de cambiar ya que una cultura no se cambia de un día para el otro. Esto puede llevar a\~nos o siglos en implementarse.