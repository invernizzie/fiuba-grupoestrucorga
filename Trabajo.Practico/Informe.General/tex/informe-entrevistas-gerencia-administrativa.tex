\section{Gerencia Administrativa: Graciela Penas}

	\paragraph{Pregunta}
	 ?`Cu\'al es su formaci\'on?  ?`Corresponde a las competencias requeridas por el cargo?  ?`Mantiene actualizados sus conociemientos?
	\paragraph{Respuesta}
	Me recib� de Licenciada en Administraci�n de Empresas y Contadora P�blica. Mi cargo excede las competencias requeridas actualmente por la empresa. Sigo haciendo cursos y actualiz�ndome.

	\paragraph{Pregunta}
	 ?`Cu\'al es su experiencia laboral?
	\paragraph{Respuesta}
	Estuve 16 a\~nos como Gerente Administrativo y Financiero en un Molino Harinero (Molino Osiris); tambi�n trabaj� para Julia Tours, de forma independiente, al igual que para Marsans, formando el departamento administrativo; y finalmente llegue a Marsans de forma efectiva.

	\paragraph{Pregunta}
	 ?`Cu\'al es su funci\'on en la empresa?
	\paragraph{Respuesta}
	Gerente Administrativo.

	\paragraph{Pregunta}
	 ?`Cu\'al es su funci\'on?  ?`Cu\'ales son sus responsabilidades?
	\paragraph{Respuesta}
	Son las relacionadas con la supervisi�n de las registraciones contables de la empresa, conciliaciones de cuentas patrimoniales y de resultado, presentaci�n de informes mensuales y trimestrales, confecci�n de balance mensual semestral y anual, pagos y presentaciones de impuestos y previsionales ante los organismos de control.

	\paragraph{Pregunta}
	 ?`Qui\'enes son sus superiores? 
	\paragraph{Respuesta}
	El Gerente General.

	\paragraph{Pregunta}
	 ?`A qui\'en le reporta?
	\paragraph{Respuesta}
	Al Gerente General o Directores Internacionales.

	\paragraph{Pregunta}
	 ?`Qui\'enes son sus subordinados?  ?`Qu\'e tareas delega sobre ellos?
	\paragraph{Respuesta}
	Actualmente por la reestructuraci�n de personal, cuento con 2 personas a cargo, a las que se le delega: contabilizaci�n de facturas de proveedores, conciliaciones bancarias, conciliaciones contables y liquidaci�n de impuestos.

	\paragraph{Pregunta}
	 ?`Delega tareas sobre personas de otras \'areas?
	\paragraph{Respuesta}
	S�, sobre el area de finanzas, por ejemplo pagos a proveedores, es decir, la conciliaci�n con los mismos e idem con cobranzas pero para la parte de deudores.

	\paragraph{Pregunta}
	 ?`Se le presentan los mismos problemas a diario o con cierta periodicidad?  ?`Cu\'ales son y c\'omo los resuelve?
	\paragraph{Respuesta}
	No, son siempre nuevos.

	\paragraph{Pregunta}
	 ?`En base a qu\'e tipo de informaci\'on realiza sus decisiones?  ?`De d\'onde la obtiene?
	\paragraph{Respuesta}
	Las decisiones son tomadas en base a la informaci�n que se saca del sistema contable.
	
	\paragraph{Pregunta}
	 ?`Qu\'e informaci\'on debe reportar?
	\paragraph{Respuesta}
	Al final de cada mes se reporta al exterior y al Director General los resultados de dicho mes, que son los ajustes de presupuesto o partidas presupuestarias y balances.

	\paragraph{Pregunta}
	 ?`Cu\'al es su relaci\'on y c\'omo se comunica con los dem\'as gerentes?
	\paragraph{Respuesta}
	Se realiza mediante los lineamientos prefijados, es decir, en el caso de esta �rea se respetan los procedimientos.

	\paragraph{Pregunta}
	 ?`De la producci\'on de qu\'e \'areas depende la concreci\'on de la propia? (Interdependencia operativa)
	\paragraph{Respuesta}
	No se depende de ning�n �rea.

	\paragraph{Pregunta}
	 ?`Considera que existe alg\'un problema con la estructura de la empresa o la organizaci\'on actual de su gerencia?  ?`Tiene la posibilidad de cambiarla?  ?`Planea hacerlo?
	\paragraph{Respuesta}
	S�, actualmente considero que es escaso el personal para realizar las tareas pertinentes del �rea. Dada la situaci�n de la empresa no cuento con la posibilidad de realizar cambios.

	\paragraph{Pregunta}
	 ?`Existe alg\'un tipo de evaluaci\'on de desempe\~{n}o?  ?`Cada cu\'anto tiempo se hace?  ?`Qu\'e aspectos se eval\'uan?  ?`Se encuentra conforme con su \'ultima evaluaci\'on recibida?
	\paragraph{Respuesta}
	No, no existe evaluaci�n.

	\paragraph{Pregunta}
	 ?`Existe alg\'un tipo de sistema de bonos?  ?`C\'omo funciona?  ?`C\'omo se determina lo que le corresponde a cada empleado?
	\paragraph{Respuesta}
	No, no existe. Hasta Enero exist�a pero se suprimi� por la situaci�n actual de la empresa.

	\paragraph{Pregunta}
	 ?`Considera adecuada la estructura de las dem\'as gerencias?  ?`Observa disfunciones?
	\paragraph{Respuesta}
	Considero que el tama�o de las �reas no se encuentra distribu�do correctamente en funci�n de la cantidad de personal.

	\paragraph{Pregunta}
	 ?`Considera que a su sector le sobra o le falta personal para realizar sus tareas?
	\paragraph{Respuesta}
	Le falta personal, como mencion� anteriormente.

	\subsection{Comentarios Adicionales}
	La empresa si bien es internacional, est� gerenciada como una PyME familiar.
	La persona a cargo de la empresa no cuenta con ninguna formaci�n profesional, no sabiendo distinguir lo que es un activo de un pasivo, devengado de percibido, ni patrimonial de financiero; haciendo muy dif�cil la interpretaci�n de un informe profesional interno o de los auditores externos, lo que resulta en decisiones bajo caprichos.
	No hay polit�cas de empresa, no hay planeamiento, tapmoco hay una integraci�n vertical. La comunicaci�n es de forma informal y no se utilizan los canones establecidos para ello.