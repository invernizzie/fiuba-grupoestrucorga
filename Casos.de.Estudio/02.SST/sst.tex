\chapter{S-S Technologies Inc.}

\section{Historia de la Empresa}
	S-S Technologies Inc. se constituy\'{o} en 1992 cuando la  compa\~{n}\'{i}a constructora y de ingenier\'{i}a integrada Sutherlaud- Schultz, de la cual formaba parte, cambi\'{o} de propietarios y los nuevos due\~{n}os vendieron la parte de SST al ex presidente de Sutherlaud- Schultz, Brock. La compa\~{n}\'{i}a era de  propiedad 100 por ciento canadiense. En \'{u}ltima instancia, SST era propiedad de su director general, Rick Brock, ex presidente de Sutherland-Schultz, y de Keith Pritchard, presidente de SST. 
	En enero de 1994 la compa\~{n}\'{i}a enfrentaba un r\'{a}pida tasa de crecimiento que se estimaba pod\'{i}a duplicarse o triplicarse en los dos a\~{n}os siguientes. En los \'{u}ltimos 3 a\~{n}os se hab\'{i}a percibido un crecimiento promedio de 33 por ciento anual y se esperaba un crecimiento de 30 a 60 o incluso 120 empleados.

	\subsection{Avance Cronol\'{o}gico de la empresa}
	\begin{enumerate}
		\item \underline{1992}. Se constituye la compa\~{n}\'{i}a al venderse la parte correspondiente a Brock. 
		\item \underline{1993}. Se perciben ingresos por 6.3 millones de d\'{o}lares.
		\item \underline{1994}. Se estima un crecimiento que duplique o triplique al del a\~{n}o anterior.
	\end{enumerate}

\section{Resumen del Funcionamiento}

	\subsection{Caracter\'{i}sticas del Sistema de Producci\'{o}n}
	\begin{itemize}
		\item La p oducci\'{o}n se divide en dos grupos: productos y sistemas integrados:
		\begin{enumerate}
			\item Productos participaba del desarrollo y venta de productos de hardware y sofware propios de la compa\~{n}\'{i}a, los cuales se vend\'{i}an por todo el mundo. Los mismos comprend\'{i}an las tarjetas simuladoras Direct-Link y el simulador PICS, los cuales peresentaban una soluci\'{o}n a un problema que ninguna otra compa\~{n}\'{i}a pod\'{i}a resolver.
			\item El grupo de sistemas integrados trabajaba en tres \'{a}reas distintas pero interrealacionadas: consultor\'{i}a, ingenier\'{i}a de sistemas y soporte a clientes brindando soluciones de calidad a problemas de hardware y software a complejos sistemas de pisos de f\'{a}bricas.
			\end{enumerate} 
		\item Ante la recesi\'{o}n, la compa\~{n}\'{i}a se ve beneficiada ya que brinda soluciones que permiten reducir los costos de automatizaci\'{o}n de las plantas y por ende los costos de producci\'{o}n.
		\item Se ten\'{i}a un enfoque de consultor\'{i}a que permit\'{i}a que quienes participaban tuvieran facultades de decisi\'{o}n y llevaba a apropiarse de los problemas y sus soluciones.
		\item Los proyectos eran asignados a individuos o equipos, seg\'{u}n el tama\~{n}o, los cuales eran autoadministrados. La responsabilidad de los proyectos reca\'{i}a en los integrantes del grupo generando un alta motivaci\'{o}n.
		\item La organizaci\'{o}n era austera, manteniendo los recursos indirectos al m\'{i}nimo.
		\item El principal recurso de la empresa lo representaba su capital humano formado de equipos muy competentes, t\'{e}cnicos y motivados. El personal t\'{e}cnico era l\'{i}der en su campo.
		\item Otro recurso era su excelente reputaci\'{o}n y relaci\'{o}n con un gran fabricante canadiense.
		\item La empresa estaba adelantada en la curva de apredizaje, habiendo enfrentado varios desaf\'{i}os, los cuales pudo resolver con \'{e}xito.
	\end{itemize}

\newpage
\section{An\'{a}lisis del caso}

	\subsection{Marco Te\'orico}

		\subsubsection{Dise\~{n}o administrativo}
			Si bien tiene un organigrama definido, la mayor parte de las relaciones que se dan en la misma son de car\'{a}cter informal. Se conforman grupos de trabajo con mecanismo coordinador de ajuste mutuo sobre los que recae la responsabilidad de las tareas que llevan a cabo. La forma de reportar no respeta el organigrama sino que se da de acuerdo a relaciones informales o costumbre.

		\subsubsection{Rasgos de pensamiento administrativo}
			La empresa presta especial atenci\'{o}n a las ciencias del comportamiento. Se busca que todos los empleados participen y sientan que todas sus necesidades se encuentran satisfechas a la vez que se valora la motivaci\'{o}n y el compromiso. Las relaciones informales cobran especial importancia con el sistema de comunicaci\'{o}n abierto que se implementa.

	\subsection{Problemas de la empresa y sus respectivas soluciones}


		\subsubsection{Falta de claridad en las metas y estrategias de la empresa}
			Los empleados no tienen en claro las metas y estrategias de la empresa. 
			Las funciones y responsabilidades de los distintos puestos resultan confusas.
			El abuso de las relaciones informales y poco respeto al organigrama genera que los empleados no reporten a quien deber\'{i}an creando caos. 
		\paragraph{Soluci\'on}
			Deber\'{i}a utilizarse un enfoque burocr\'{a}tico a fin de definir la autoridad, funciones y responsabilidades de los distintos puestos y dejar en claro la jerarqu\'{i}a de los cargos. 
			Aquellas personas en puestos de autoridad deber\'{i}an exigir que se cumplan estas relaciones. 
			Es necesario designar formalmente un gerente para el Grupo de Productos y otro para el Grupo de Sistemas Integrados. 
			Esto puede traer un conflicto interno dentro del grupo GSI  pero es necesario para resolver las confusiones que tienen los integrantes del grupo en lo que respecta a la autoridad de Shwarz respecto a Ojala.
		
		\subsubsection{Falta de espacio f\'isico}
			La compa\~{n}\'{i}a hab\'{i}a crecido tanto que resultaban insuficientes las instalaciones que compart\'{i}a con otras empresas. Como no se quer\'{i}a separar al personal, se agreg\'{o} un remolque para ubicar al personal adicional.
		\paragraph{Soluci\'on}
			Una alternativa es la b\'{u}squeda de un edificio con mayor disponibilidad de instalaciones a fin de que todos los empleados puedan trabajar en un lugar c\'{o}modo sin que sea necesaria la divisi\'{o}n del mismo. 
			Existe una segunda alternativa que es reubicar a la otra empresa del dueo de SST, SAF o esperar a la culminaci\'{o}n  o rescindir el pr\'{e}stamo con la empresa Wilson Gas. 
			La ventaja de la primera alternativa es que la empresa no tendr\'{i}a que rescindir ning\'{u}n contrato ni tratar con otra empresa. 
			Adem\'{a}s se podr\'{i}a buscar un espacio que se adecue mejor a determinadas necesidades que despu\'{e}s de a\~{n}os en la producci\'{o}n, se identifican f\'{a}cilmente. 
			Las ventajas de la segunda alternativa es que la empresa no atravesar\'{i}a una mudanza en un momento en el cual, como muestra el caso, se espera una escasez de recursos y que como consecuencia se necesitar\'{i}a que todo recurso existente este abocado a satisfacer las necesidades del mercado sin perder tiempo en una mudanza.

		\subsubsection{No hay una figura que realice una tarea de recursos humanos}
			No se registran las evaluaciones de desempe\~{n}o y no existe una figura a la cual se le puedan hacer consultas referidas a los l\'{i}mites salariales de los diversos puestos.
		\paragraph{Soluci\'on}
			Creaci\'{o}n de un \'area de recursos humanos para que pueda guardar registro de las evaluaciones de desempe\~{n}o y que adem\'{a}s sirva para que los empleados puedan hacer consultas no t\'{e}cnicas sin sentirse inc\'{o}modos por tener que realizar la consultar a un superior.  

		\subsubsection{Falta de incentivo salarial a los empleados}
			Conflicto generado por ausencia de bonos y compensaciones. Los empleados no tienen un incentivo o un dinero extra por el trabajo realizado, con lo cual esto puede generar conflictos futuros (y no tan futuros) con los empleados.
		\paragraph{Soluci\'on}
			Se propone que la empresa realice evaluaciones individuales de performance del personal. 
			Para esto es necesario que se defina y respete un organigrama a fin de que quede en claro qui\'{e}n estar\'{a} siendo evaluado por qui\'{e}n. 
			Los mismos solo deberan premiar un excelente desempe\~{n} y preferentemente ser entregados a equipos de personal a fin de fomentar la cooperaci\'{o}n y limitar la competencia interpersonal. 
			Tambi\'{e}n debe considerarse entregar premios adicionales por contribuciones excepcionales.
			Los gerentes de cada \'{a}rea notificar\'{i}an al \'{a}rea de recursos humanos de los resultados de las evaluaciones t\'{e}cnicas para que este \'{a}rea sea la encargada de entregar los bonos y premios.
			Sera una funci\'{o}n del \'{a}rea de recursos humanos estudiar el estado econmico de cada empleado para analizar si es viable dar en forma de bonificaci\'{o}n acciones de la empresa para poder aumentar el sentimiento de pertenencia que siente cada empleado.


		\subsubsection{R\'apido crecimiento}
			Inminente crecimiento de la empresa en el corto plazo. La empresa en un per\'iodo corto de tiempo implement\'o un crecimiento importante, el 				cual fue demasiado r\'apido e hizo que se deban adaptar muchas cosas, algunas en forma eficiente y otras en forma ineficiente.
		\paragraph{Soluci\'on}
			Considerando que la empresa cuenta con mano de obra muy calificada es necesario ir contratando empleados para poder capacitarlos adecuadamente. 
			Teniendo en cuenta que la capacitaci\'{o}n implica que los empleados actuales destinen horas de producci\'{o}n e investigaci\'{o}n a la capacitaci\'{o}n de los nuevos empleados, la contrataci\'{o}n de personal deber\'{a} hacerse de a poco para que la producci\'{o}n no se vea comprometida.